\documentclass{IOS-Book-Article}
% generated by Madoko, version 0.8.1-beta
% data-line={1}

\usepackage{madoko}


\begin{document}
\begin{frontmatter}              % The preamble begins here.
%\pretitle{Pretitle}
\title{Deconstructing Dynamic Symbolic Execution}
%\runningtitle{IOS Press Style Sample}
%\subtitle{Subtitle}
\author[A]{\fnms{Thomas} \snm{Ball}}
and 
\author[B]{\fnms{Jakub} \snm{Daniel}}

\runningauthor{Thomas Ball et al.}
\address[A]{Microsoft Research}
\address[B]{Charles University}\begin{mdDiv}[class={abstract},elem={abstract},data-line={39}]%
\begin{mdP}[data-line={40}]%
% data-line={40}
{}Dynamic symbolic execution (DSE) is a well-known technique
for automatically generating tests to achieve higher levels
of coverage in a program. Two keys ideas of DSE are
to: (1) seed symbolic execution by executing a program on an
initial input; (2) using concrete values from the program
execution in place of symbolic expressions whenever symbolic
reasoning is hard or not desired. We describe
DSE for a simple core language and then present
a minimalist implementation of DSE for Python (in Python) 
that follows this basic recipe. The code is available 
at https://www.github.com/thomasjball/PyExZ3/ (tagged % data-line={50}
{}{\textquotedblleft}v1.0{\textquotedblright}% data-line={50}
{}) 
and has been designed to make it easy to experiment with and
extend.%
\end{mdP}%%
\end{mdDiv}%
\begin{keyword}
Symbolic Execution, Automatic Test Generation, White-box Testing, Automated 
Theorem Provers
\end{keyword}
\end{frontmatter}
\thispagestyle{empty}
\pagestyle{empty}\mdHxx[id=sec-intro,label={[1]\{.heading-label\}},toc={},data-line={70},caption={[[1]\{.heading-label\}.{\enspace}]\{.heading-before\}Introduction},bookmark={1.{\enspace}Introduction}]{% data-line={70}
{}\mdSpan[class={heading-before}]{\mdSpan[class={heading-label}]{1}.{\enspace}}% data-line={70}
{}Introduction}% data-line={73}
\defcommand{\mathkw}[1]{\textbf{#1}}\begin{mdP}[data-line={78}]%
% data-line={78}
{}Static, path-based symbolic execution explores one control-flow path
at a time through a (sequential) program % data-line={79}
{}\mdSpan[class={math-inline},elem={math-inline}]{$P$}% data-line={79}
{}, using an automated theorem
prover (ATP) to determine if the current path % data-line={80}
{}\mdSpan[class={math-inline},elem={math-inline}]{$p$}% data-line={80}
{} is feasible% data-line={80}
{}{\mdNbsp}\mdSpan[class={citations},target-element={bibitem}]{[\mdA[class={bibref,localref},target-element={bibitem}]{clarke76}{}{\mdSpan[class={bibitem-label}]{4}}, \mdA[class={bibref,localref},target-element={bibitem}]{king76}{}{\mdSpan[class={bibitem-label}]{11}}]}% data-line={80}
{}. 
Ideally, symbolic execution of a path % data-line={81}
{}\mdSpan[class={math-inline},elem={math-inline}]{$p$}% data-line={81}
{} through program
% data-line={82}
{}\mdSpan[class={math-inline},elem={math-inline}]{$P$}% data-line={82}
{} yields a logic formula % data-line={82}
{}\mdSpan[class={math-inline},elem={math-inline}]{$\phi_p$}% data-line={82}
{} that describes the set of inputs % data-line={82}
{}\mdSpan[class={math-inline},elem={math-inline}]{$I$}% data-line={82}
{} (possibly empty)
to program % data-line={83}
{}\mdSpan[class={math-inline},elem={math-inline}]{$P$}% data-line={83}
{} such that for any % data-line={83}
{}\mdSpan[class={math-inline},elem={math-inline}]{$i \in I$}% data-line={83}
{}, the execution % data-line={83}
{}\mdSpan[class={math-inline},elem={math-inline}]{$P(i)$}% data-line={83}
{} follows path % data-line={83}
{}\mdSpan[class={math-inline},elem={math-inline}]{$p$}% data-line={83}
{}.%
\end{mdP}%
\begin{mdP}[class={indent},data-line={85}]%
% data-line={85}
{}If the formula % data-line={85}
{}\mdSpan[class={math-inline},elem={math-inline}]{$\phi_p$}% data-line={85}
{} is unsatisfiable then % data-line={85}
{}\mdSpan[class={math-inline},elem={math-inline}]{$I$}% data-line={85}
{} is empty and so path % data-line={85}
{}\mdSpan[class={math-inline},elem={math-inline}]{$p$}% data-line={85}
{} is not feasible; 
if the formula is satisfiable then % data-line={86}
{}\mdSpan[class={math-inline},elem={math-inline}]{$I$}% data-line={86}
{} is not empty and so path % data-line={86}
{}\mdSpan[class={math-inline},elem={math-inline}]{$p$}% data-line={86}
{} is feasible.
In this case, a model of % data-line={87}
{}\mdSpan[class={math-inline},elem={math-inline}]{$\phi_p$}% data-line={87}
{} provides a witness % data-line={87}
{}\mdSpan[class={math-inline},elem={math-inline}]{$i \in I$}% data-line={87}
{}.  Thus, an ATP
that can provide such models can be used in conjunction with
symbolic execution to automatically generate tests to cover paths
in a program. Combined with a search strategy, one gets, in the limit,
an exhaustive white-box testing procedure, for which there are many
applications% data-line={92}
{}{\mdNbsp}\mdSpan[class={citations},target-element={bibitem}]{[\mdA[class={bibref,localref},target-element={bibitem}]{cadars13}{}{\mdSpan[class={bibitem-label}]{2}}, \mdA[class={bibref,localref},target-element={bibitem}]{cadargpde06}{}{\mdSpan[class={bibitem-label}]{3}}, \mdA[class={bibref,localref},target-element={bibitem}]{godefroidlm12}{}{\mdSpan[class={bibitem-label}]{9}}]}% data-line={92}
{}.%
\end{mdP}%
\begin{mdP}[class={indent},data-line={94}]%
% data-line={94}
{}The formula % data-line={94}
{}\mdSpan[class={math-inline},elem={math-inline}]{$\phi_p$}% data-line={94}
{} is called a % data-line={94}
{}\mdEm{path-condition}% data-line={94}
{} of the path % data-line={94}
{}\mdSpan[class={math-inline},elem={math-inline}]{$p$}% data-line={94}
{}. 
We will see that a given path % data-line={95}
{}\mdSpan[class={math-inline},elem={math-inline}]{$p$}% data-line={95}
{} can induce many different path-conditions.
A path-condition % data-line={96}
{}\mdSpan[class={math-inline},elem={math-inline}]{$\psi_p$}% data-line={96}
{} for path % data-line={96}
{}\mdSpan[class={math-inline},elem={math-inline}]{$p$}% data-line={96}
{} is % data-line={96}
{}\mdEm{sound}% data-line={96}
{} if 
every input assignment satisfying % data-line={97}
{}\mdSpan[class={math-inline},elem={math-inline}]{$\psi_p$}% data-line={97}
{} defines an
execution of program % data-line={98}
{}\mdSpan[class={math-inline},elem={math-inline}]{$P$}% data-line={98}
{} that follows path % data-line={98}
{}\mdSpan[class={math-inline},elem={math-inline}]{$p$}% data-line={98}
{}{\mdNbsp}\mdSpan[class={citations},target-element={bibitem}]{[\mdA[class={bibref,localref},target-element={bibitem}]{godefroid11}{}{\mdSpan[class={bibitem-label}]{7}}]}% data-line={98}
{}. 
By its definition, the formula  % data-line={99}
{}\mdSpan[class={math-inline},elem={math-inline}]{$\phi_p$}% data-line={99}
{} is sound and the best representation of % data-line={99}
{}\mdSpan[class={math-inline},elem={math-inline}]{$p$}% data-line={99}
{}
(as for all sound path-conditions % data-line={100}
{}\mdSpan[class={math-inline},elem={math-inline}]{$\psi_p$}% data-line={100}
{}, we have that % data-line={100}
{}\mdSpan[class={math-inline},elem={math-inline}]{$\psi_p \implies \phi_p$}% data-line={100}
{}).
In practice, we attempt to compute sound under-approximations of % data-line={101}
{}\mdSpan[class={math-inline},elem={math-inline}]{$\phi_p$}% data-line={101}
{} 
such as % data-line={102}
{}\mdSpan[class={math-inline},elem={math-inline}]{$\psi_p$}% data-line={102}
{}. However, we also find it necessary (and useful) to 
compute unsound path-conditions.%
\end{mdP}%
\begin{mdP}[class={indent},data-line={105}]%
% data-line={105}
{}A path-condition can be translated into the input
language of an ATP, such as% data-line={106}
{}{\mdNbsp}\mdA{http://z3.codeplex.org/}{}{Z3}% data-line={106}
{}\mdSpan[class={citations},target-element={bibitem}]{[\mdA[class={bibref,localref},target-element={bibitem}]{demourab08}{}{\mdSpan[class={bibitem-label}]{5}}]}% data-line={106}
{}, which provides an answer
of % data-line={107}
{}{\textquotedblleft}unsatisfiable{\textquotedblright}% data-line={107}
{}, % data-line={107}
{}{\textquotedblleft}satisfiable{\textquotedblright}% data-line={107}
{} or % data-line={107}
{}{\textquotedblleft}unknown{\textquotedblright}% data-line={107}
{}, due to theoretical or practical
limitations in automatically deciding satisfiability of various logics.
In the case that the ATP is able to prove % data-line={109}
{}{\textquotedblleft}satisfiable{\textquotedblright}% data-line={109}
{} we can query it for 
satisfying model in order to generate test inputs. A path-condition 
for % data-line={111}
{}\mdSpan[class={math-inline},elem={math-inline}]{$p$}% data-line={111}
{} can be thought of as function from a
program% data-line={112}
{}{'}% data-line={112}
{}s primary inputs to a Boolean output representing whether
or not % data-line={113}
{}\mdSpan[class={math-inline},elem={math-inline}]{$p$}% data-line={113}
{} is executed under a given input. Thus, we are asking
the ATP to invert a function when we ask it to decide
the satisfiability/unsatisfiability of a path-condition.%
\end{mdP}%
\begin{mdP}[class={indent},data-line={117}]%
% data-line={117}
{}The static translation of a path % data-line={117}
{}\mdSpan[class={math-inline},elem={math-inline}]{$p$}% data-line={117}
{} through a program % data-line={117}
{}\mdSpan[class={math-inline},elem={math-inline}]{$P$}% data-line={117}
{} into 
the most precise path-condition % data-line={118}
{}\mdSpan[class={math-inline},elem={math-inline}]{$\phi_p$}% data-line={118}
{} is not a simple task, as 
programming languages and their semantics are very complex.
Completely characterizing the set of inputs % data-line={120}
{}\mdSpan[class={math-inline},elem={math-inline}]{$I$}% data-line={120}
{} that follow
path % data-line={121}
{}\mdSpan[class={math-inline},elem={math-inline}]{$p$}% data-line={121}
{} means providing a symbolic interpretation of 
every operation in the language so that the
ATP can reason about it. For example, consider a method call in Python. 
Python% data-line={124}
{}{'}% data-line={124}
{}s algorithm for method resolution order (see% data-line={124}
{}{\mdNbsp}\mdA{https://www.python.org/download/releases/2.3/mro/}{}{MRO}% data-line={124}
{})
depends on the inheritance hierarchy of the program, a directed, 
acyclic graph that can evolve during program execution. Symbolically
encoding Python% data-line={127}
{}{'}% data-line={127}
{}s method resolution order is possible but non-trivial.
There are other reasons it is hard or undesirable to symbolically 
execute various operations, as will be in explained in detail later.%
\end{mdP}%
\mdHxxx[id=sec-dynamic-symbolic-execution,label={[1.1]\{.heading-label\}},toc={},data-line={131},caption={[[1.1]\{.heading-label\}.{\enspace}]\{.heading-before\}Dynamic symbolic execution},bookmark={1.1.{\enspace}Dynamic symbolic execution}]{% data-line={131}
{}\mdSpan[class={heading-before}]{\mdSpan[class={heading-label}]{1.1}.{\enspace}}% data-line={131}
{}Dynamic symbolic execution}\begin{mdP}[data-line={133}]%
% data-line={133}
{}\mdEm{Dynamic}% data-line={133}
{} symbolic execution (DSE) is a form of path-based
symbolic execution based on two insights. First, the approach 
starts by executing program % data-line={135}
{}\mdSpan[class={math-inline},elem={math-inline}]{$P$}% data-line={135}
{} on
some input % data-line={136}
{}\mdSpan[class={math-inline},elem={math-inline}]{$i$}% data-line={136}
{}, seeding the symbolic execution process
with a feasible path% data-line={137}
{}{\mdNbsp}\mdSpan[class={citations},target-element={bibitem}]{[\mdA[class={bibref,localref},target-element={bibitem}]{gupta00}{}{\mdSpan[class={bibitem-label}]{10}}, \mdA[class={bibref,localref},target-element={bibitem}]{korel90}{}{\mdSpan[class={bibitem-label}]{12}}, \mdA[class={bibref,localref},target-element={bibitem}]{korel92}{}{\mdSpan[class={bibitem-label}]{13}}]}% data-line={137}
{}. 
Second,  DSE
uses concrete values from the execution % data-line={139}
{}\mdSpan[class={math-inline},elem={math-inline}]{$P(i)$}% data-line={139}
{} in place of symbolic expressions 
whenever symbolic reasoning is not possible or desired% data-line={140}
{}{\mdNbsp}\mdSpan[class={citations},target-element={bibitem}]{[\mdA[class={bibref,localref},target-element={bibitem}]{cadare05}{}{\mdSpan[class={bibitem-label}]{1}}, \mdA[class={bibref,localref},target-element={bibitem}]{godefroidks05}{}{\mdSpan[class={bibitem-label}]{8}}]}% data-line={140}
{}.
The major benefit of DSE is to
simplify the construction of a symbolic execution tool by
leveraging the concrete execution behavior (given by
actually running the program).
As DSE combines both 
concrete and symbolic reasoning, it also has been called % data-line={146}
{}{\textquotedblleft}concolic{\textquotedblright}% data-line={146}
{} 
execution% data-line={147}
{}{\mdNbsp}\mdSpan[class={citations},target-element={bibitem}]{[\mdA[class={bibref,localref},target-element={bibitem}]{senacav06}{}{\mdSpan[class={bibitem-label}]{14}}]}% data-line={147}
{}.%
\end{mdP}%
\begin{mdDiv}[class={figure,align-center},id=fig-dse,label={[1]\{.figure-label\}},elem={figure},toc={tof},toc-line={[1]\{.figure-label\}. Pseudo-code for dynamic symbolic execution},caption={Pseudo-code for dynamic symbolic execution},data-line={149}]%
\begin{mdPre}[class={para-block,pre-fenced,pre-fenced3,language-python,lang-python,python,highlighted},language={python},data-line={150},data-line-code={151}]%
\mdPrecode[data-line={151}]{{\preindent{2}}\mdToken{Identifier,Python}{i}{\prespace{1}}\mdToken{Keyword,Python}{=}{\prespace{1}}\mdToken{Identifier,Python}{an}{\prespace{1}}\mdToken{Identifier,Python}{input}{\prespace{1}}\mdToken{Identifier,Python}{to}{\prespace{1}}\mdToken{Identifier,Python}{program}{\prespace{1}}\mdToken{Constructor,Identifier,Python}{P}\prebr{}
{\preindent{2}}\mdToken{Keyword,Python}{while}{\prespace{1}}\mdToken{Identifier,Python}{defined}\mdToken{Delimiter,Parenthesis,Python,BracketOpen}{(}\mdToken{Identifier,Python}{i}\mdToken{Delimiter,Parenthesis,Python,BracketClose}{)}\mdToken{Keyword,Python,BracketOpen}{:}\prebr{}
{\preindent{5}}\mdToken{Identifier,Python}{p}{\prespace{1}}\mdToken{Keyword,Python}{=}{\prespace{1}}\mdToken{Identifier,Python}{path}{\prespace{1}}\mdToken{Identifier,Python}{covered}{\prespace{1}}\mdToken{Identifier,Python}{by}{\prespace{1}}\mdToken{Identifier,Python}{execution}{\prespace{1}}\mdToken{Constructor,Identifier,Python}{P}\mdToken{Delimiter,Parenthesis,Python,BracketOpen}{(}\mdToken{Identifier,Python}{i}\mdToken{Delimiter,Parenthesis,Python,BracketClose}{)}\prebr{}
{\preindent{5}}\mdToken{Identifier,Python}{cond}{\prespace{1}}\mdToken{Keyword,Python}{=}{\prespace{1}}\mdToken{Identifier,Python}{pathCondition}\mdToken{Delimiter,Parenthesis,Python,BracketOpen}{(}\mdToken{Identifier,Python}{p}\mdToken{Delimiter,Parenthesis,Python,BracketClose}{)}\prebr{}
{\preindent{5}}\mdToken{Identifier,Python}{s}{\prespace{1}}\mdToken{Keyword,Python}{=}{\prespace{1}}\mdToken{Constructor,Identifier,Python}{ATP}\mdToken{Delimiter,Parenthesis,Python,BracketOpen}{(}\mdToken{Namespace,Identifier,Python}{Not}\mdToken{Delimiter,Parenthesis,Python,BracketOpen}{(}\mdToken{Identifier,Python}{cond}\mdToken{Delimiter,Parenthesis,Python,BracketClose}{)}\mdToken{Delimiter,Parenthesis,Python,BracketClose}{)}\prebr{}
{\preindent{5}}\mdToken{Identifier,Python}{i}{\prespace{1}}\mdToken{Keyword,Python}{=}{\prespace{1}}\mdToken{Identifier,Python}{s}\mdToken{Delimiter,Python}{.}\mdToken{Identifier,Python}{model}\mdToken{Delimiter,Parenthesis,Python,BracketOpen}{(}\mdToken{Delimiter,Parenthesis,Python,BracketClose}{)}}%
\end{mdPre}%
\mdHr[class={figureline,madoko},data-line={158}]{}\begin{mdDiv}[data-line={159}]%
% data-line={159}
{}\mdSpan[class={figure-caption}]{\mdSpan[class={caption-before}]{\mdStrong{Figure{\mdNbsp}\mdSpan[class={figure-label}]{1}.} }Pseudo-code for dynamic symbolic execution}% data-line={159}
{}%
\end{mdDiv}%%
\end{mdDiv}%
\begin{mdP}[class={indent},data-line={160}]%
% data-line={160}
{}The pseudo-code of Figure% data-line={160}
{}{\mdNbsp}\mdA[class={localref},target-element={figure}]{fig-dse}{}{\mdSpan[class={figure-label}]{1}}% data-line={160}
{} shows the high level process
of DSE. The variable % data-line={161}
{}\mdCode[class={code,code1,language-python,lang-python,python,highlighted},language={python}]{\mdToken{Identifier,Python}{i}}% data-line={161}
{} represents an input
to program % data-line={162}
{}\mdCode[class={code,code1,language-python,lang-python,python,highlighted},language={python}]{\mdToken{Constructor,Identifier,Python}{P}}% data-line={162}
{}. Execution of program % data-line={162}
{}\mdCode[class={code,code1,language-python,lang-python,python,highlighted},language={python}]{\mdToken{Constructor,Identifier,Python}{P}}% data-line={162}
{} on the input % data-line={162}
{}\mdCode[class={code,code1,language-python,lang-python,python,highlighted},language={python}]{\mdToken{Identifier,Python}{i}}% data-line={162}
{}
traces  a path % data-line={163}
{}\mdCode[class={code,code1,language-python,lang-python,python,highlighted},language={python}]{\mdToken{Identifier,Python}{p}}% data-line={163}
{}, from which
a logical formula % data-line={164}
{}\mdCode[class={code,code1,language-python,lang-python,python,highlighted},language={python}]{\mdToken{Identifier,Python}{pathCondition}\mdToken{Delimiter,Parenthesis,Python,BracketOpen}{(}\mdToken{Identifier,Python}{p}\mdToken{Delimiter,Parenthesis,Python,BracketClose}{)}}% data-line={164}
{} is constructed.
Finally, the ATP is called with the negation of the path-condition
to find a new input (that hopefully will cover a new path).  This
pseudo-code elides a number of details that we will deal with later.%
\end{mdP}%
\begin{mdDiv}[class={figure,align-center},id=fig-easy-dse,label={[2]\{.figure-label\}},elem={figure},toc={tof},toc-line={[2]\{.figure-label\}. Easy example: computing the maximum of four numbers in Python.},caption={Easy example: computing the maximum of four numbers in Python.},data-line={169}]%
\begin{mdPre}[class={para-block,pre-fenced,pre-fenced3,language-python,lang-python,python,highlighted},language={python},data-line={170},data-line-code={171}]%
\mdPrecode[data-line={171}]{\mdToken{Keyword,Python}{def}{\prespace{1}}\mdToken{Identifier,Python}{max2}\mdToken{Delimiter,Parenthesis,Python,BracketOpen}{(}\mdToken{Identifier,Python}{s}\mdToken{Delimiter,Python}{,}\mdToken{Identifier,Python}{t}\mdToken{Delimiter,Parenthesis,Python,BracketClose}{)}\mdToken{Keyword,Python,BracketOpen}{:}\prebr{}
{\preindent{4}}\mdToken{Keyword,Python}{if}{\prespace{1}}\mdToken{Delimiter,Parenthesis,Python,BracketOpen}{(}\mdToken{Identifier,Python}{s}{\prespace{1}}\mdToken{Operator,Python}{{\textless}}{\prespace{1}}\mdToken{Identifier,Python}{t}\mdToken{Delimiter,Parenthesis,Python,BracketClose}{)}\mdToken{Keyword,Python,BracketOpen}{:}\prebr{}
{\preindent{8}}\mdToken{Keyword,Python}{return}{\prespace{1}}\mdToken{Identifier,Python}{t}\prebr{}
{\preindent{4}}\mdToken{Keyword,Python}{else}\mdToken{Keyword,Python,BracketOpen}{:}\prebr{}
{\preindent{8}}\mdToken{Keyword,Python}{return}{\prespace{1}}\mdToken{Identifier,Python}{s}\prebr{}
\prebr{}
\mdToken{Keyword,Python}{def}{\prespace{1}}\mdToken{Identifier,Python}{max4}\mdToken{Delimiter,Parenthesis,Python,BracketOpen}{(}\mdToken{Identifier,Python}{a}\mdToken{Delimiter,Python}{,}\mdToken{Identifier,Python}{b}\mdToken{Delimiter,Python}{,}\mdToken{Identifier,Python}{c}\mdToken{Delimiter,Python}{,}\mdToken{Identifier,Python}{d}\mdToken{Delimiter,Parenthesis,Python,BracketClose}{)}\mdToken{Keyword,Python,BracketOpen}{:}\prebr{}
{\preindent{4}}\mdToken{Keyword,Python}{return}{\prespace{1}}\mdToken{Identifier,Python}{max2}\mdToken{Delimiter,Parenthesis,Python,BracketOpen}{(}\mdToken{Identifier,Python}{max2}\mdToken{Delimiter,Parenthesis,Python,BracketOpen}{(}\mdToken{Identifier,Python}{a}\mdToken{Delimiter,Python}{,}\mdToken{Identifier,Python}{b}\mdToken{Delimiter,Parenthesis,Python,BracketClose}{)}\mdToken{Delimiter,Python}{,}\mdToken{Identifier,Python}{max2}\mdToken{Delimiter,Parenthesis,Python,BracketOpen}{(}\mdToken{Identifier,Python}{c}\mdToken{Delimiter,Python}{,}\mdToken{Identifier,Python}{d}\mdToken{Delimiter,Parenthesis,Python,BracketClose}{)}\mdToken{Delimiter,Parenthesis,Python,BracketClose}{)}}%
\end{mdPre}%
\mdHr[class={figureline,madoko},data-line={180}]{}\begin{mdDiv}[data-line={181}]%
% data-line={181}
{}\mdSpan[class={figure-caption}]{\mdSpan[class={caption-before}]{\mdStrong{Figure{\mdNbsp}\mdSpan[class={figure-label}]{2}.} }Easy example: computing the maximum of four numbers in Python.}% data-line={181}
{}%
\end{mdDiv}%%
\end{mdDiv}%
\begin{mdP}[class={indent,para-continue},data-line={182}]%
% data-line={182}
{}Consider the  Python function % data-line={182}
{}\mdCode[class={code,code1,language-python,lang-python,python,highlighted},language={python}]{\mdToken{Identifier,Python}{max4}}% data-line={182}
{} in Figure% data-line={182}
{}{\mdNbsp}\mdA[class={localref},target-element={figure}]{fig-easy-dse}{}{\mdSpan[class={figure-label}]{2}}% data-line={182}
{},
which computes the maximum of four numbers via three calls to
the function % data-line={184}
{}\mdCode[class={code,code1,language-python,lang-python,python,highlighted},language={python}]{\mdToken{Identifier,Python}{max2}}% data-line={184}
{}. Suppose we execute % data-line={184}
{}\mdCode[class={code,code1,language-python,lang-python,python,highlighted},language={python}]{\mdToken{Identifier,Python}{max4}}% data-line={184}
{} with values
of zero for all four arguments.  In this case, the 
execution path % data-line={186}
{}\mdSpan[class={math-inline},elem={math-inline}]{$p$}% data-line={186}
{} contains three comparisons (in the order % data-line={186}
{}\mdCode[class={code,code1,language-python,lang-python,python,highlighted},language={python}]{\mdToken{Delimiter,Parenthesis,Python,BracketOpen}{(}\mdToken{Identifier,Python}{a}{\prespace{1}}\mdToken{Operator,Python}{{\textless}}{\prespace{1}}\mdToken{Identifier,Python}{b}\mdToken{Delimiter,Parenthesis,Python,BracketClose}{)}}% data-line={186}
{}, 
% data-line={187}
{}\mdCode[class={code,code1,language-python,lang-python,python,highlighted},language={python}]{\mdToken{Delimiter,Parenthesis,Python,BracketOpen}{(}\mdToken{Identifier,Python}{c}{\prespace{1}}\mdToken{Operator,Python}{{\textless}}{\prespace{1}}\mdToken{Identifier,Python}{d}\mdToken{Delimiter,Parenthesis,Python,BracketClose}{)}}% data-line={187}
{}, % data-line={187}
{}\mdCode[class={code,code1,language-python,lang-python,python,highlighted},language={python}]{\mdToken{Delimiter,Parenthesis,Python,BracketOpen}{(}\mdToken{Identifier,Python}{a}{\prespace{1}}\mdToken{Operator,Python}{{\textless}}{\prespace{1}}\mdToken{Identifier,Python}{c}\mdToken{Delimiter,Parenthesis,Python,BracketClose}{)}}% data-line={187}
{}), all of which evaluate false.
Thus, the path-condition for path % data-line={188}
{}\mdSpan[class={math-inline},elem={math-inline}]{$p$}% data-line={188}
{} is % data-line={188}
{}\mdCode[class={code,code1,language-python,lang-python,python,highlighted},language={python}]{\mdToken{Delimiter,Parenthesis,Python,BracketOpen}{(}\mdToken{Keyword,Python}{not}\mdToken{Delimiter,Parenthesis,Python,BracketOpen}{(}\mdToken{Identifier,Python}{a}\mdToken{Operator,Python}{{\textless}}\mdToken{Identifier,Python}{b}\mdToken{Delimiter,Parenthesis,Python,BracketClose}{)}{\prespace{1}}\mdToken{Keyword,Python}{and}{\prespace{1}}\mdToken{Keyword,Python}{not}\mdToken{Delimiter,Parenthesis,Python,BracketOpen}{(}\mdToken{Identifier,Python}{c}\mdToken{Operator,Python}{{\textless}}\mdToken{Identifier,Python}{d}\mdToken{Delimiter,Parenthesis,Python,BracketClose}{)}{\prespace{1}}\mdToken{Keyword,Python}{and}{\prespace{1}}\mdToken{Keyword,Python}{not}\mdToken{Delimiter,Parenthesis,Python,BracketOpen}{(}\mdToken{Identifier,Python}{a}{\prespace{1}}\mdToken{Operator,Python}{{\textless}}{\prespace{1}}\mdToken{Identifier,Python}{c}\mdToken{Delimiter,Parenthesis,Python,BracketClose}{)}\mdToken{Delimiter,Parenthesis,Python,BracketClose}{)}}% data-line={188}
{}.
Negating this condition yields % data-line={189}
{}\mdCode[class={code,code1,language-python,lang-python,python,highlighted},language={python}]{\mdToken{Delimiter,Parenthesis,Python,BracketOpen}{(}\mdToken{Delimiter,Parenthesis,Python,BracketOpen}{(}\mdToken{Identifier,Python}{a}\mdToken{Operator,Python}{{\textless}}\mdToken{Identifier,Python}{b}\mdToken{Delimiter,Parenthesis,Python,BracketClose}{)}{\prespace{1}}\mdToken{Keyword,Python}{or}{\prespace{1}}\mdToken{Delimiter,Parenthesis,Python,BracketOpen}{(}\mdToken{Identifier,Python}{c}\mdToken{Operator,Python}{{\textless}}\mdToken{Identifier,Python}{d}\mdToken{Delimiter,Parenthesis,Python,BracketClose}{)}{\prespace{1}}\mdToken{Keyword,Python}{or}{\prespace{1}}\mdToken{Delimiter,Parenthesis,Python,BracketOpen}{(}\mdToken{Identifier,Python}{a}\mdToken{Operator,Python}{{\textless}}\mdToken{Identifier,Python}{c}\mdToken{Delimiter,Parenthesis,Python,BracketClose}{)}\mdToken{Delimiter,Parenthesis,Python,BracketClose}{)}}% data-line={189}
{}.
Taking the execution ordering of the three comparisons into account, we
derive three expressions from the negated path-condition to generate
new inputs that will explore execution prefixes of path % data-line={192}
{}\mdSpan[class={math-inline},elem={math-inline}]{$p$}% data-line={192}
{} of increasing length:%
\end{mdP}%
\begin{mdUl}[class={list-star,compact},data-line={194}]%
\begin{mdLi}[data-line={194}]%
% data-line={194}
{}\mdEm{length 0}% data-line={194}
{}: % data-line={194}
{}\mdCode[class={code,code1,language-python,lang-python,python,highlighted},language={python}]{\mdToken{Delimiter,Parenthesis,Python,BracketOpen}{(}\mdToken{Identifier,Python}{a}\mdToken{Operator,Python}{{\textless}}\mdToken{Identifier,Python}{b}\mdToken{Delimiter,Parenthesis,Python,BracketClose}{)}}% data-line={194}
{}%
\end{mdLi}%
\begin{mdLi}[data-line={195}]%
% data-line={195}
{}\mdEm{length 1}% data-line={195}
{}: % data-line={195}
{}\mdCode[class={code,code1,language-python,lang-python,python,highlighted},language={python}]{\mdToken{Keyword,Python}{not}{\prespace{1}}\mdToken{Delimiter,Parenthesis,Python,BracketOpen}{(}\mdToken{Identifier,Python}{a}\mdToken{Operator,Python}{{\textless}}\mdToken{Identifier,Python}{b}\mdToken{Delimiter,Parenthesis,Python,BracketClose}{)}{\prespace{1}}\mdToken{Keyword,Python}{and}{\prespace{1}}\mdToken{Delimiter,Parenthesis,Python,BracketOpen}{(}\mdToken{Identifier,Python}{c}\mdToken{Operator,Python}{{\textless}}\mdToken{Identifier,Python}{d}\mdToken{Delimiter,Parenthesis,Python,BracketClose}{)}}% data-line={195}
{}%
\end{mdLi}%
\begin{mdLi}[data-line={196}]%
% data-line={196}
{}\mdEm{length 2}% data-line={196}
{}: % data-line={196}
{}\mdCode[class={code,code1,language-python,lang-python,python,highlighted},language={python}]{\mdToken{Keyword,Python}{not}{\prespace{1}}\mdToken{Delimiter,Parenthesis,Python,BracketOpen}{(}\mdToken{Identifier,Python}{a}\mdToken{Operator,Python}{{\textless}}\mdToken{Identifier,Python}{b}\mdToken{Delimiter,Parenthesis,Python,BracketClose}{)}{\prespace{1}}\mdToken{Keyword,Python}{and}{\prespace{1}}\mdToken{Keyword,Python}{not}{\prespace{1}}\mdToken{Delimiter,Parenthesis,Python,BracketOpen}{(}\mdToken{Identifier,Python}{c}\mdToken{Operator,Python}{{\textless}}\mdToken{Identifier,Python}{d}\mdToken{Delimiter,Parenthesis,Python,BracketClose}{)}{\prespace{1}}\mdToken{Keyword,Python}{and}{\prespace{1}}\mdToken{Delimiter,Parenthesis,Python,BracketOpen}{(}\mdToken{Identifier,Python}{a}\mdToken{Operator,Python}{{\textless}}\mdToken{Identifier,Python}{c}\mdToken{Delimiter,Parenthesis,Python,BracketClose}{)}}% data-line={196}
{}%
\end{mdLi}%%
\end{mdUl}%
\begin{mdP}[data-line={198}]%
% data-line={198}
{}The purpose of taking execution order into account should be clear, as the
comparison % data-line={199}
{}\mdCode[class={code,code1,language-python,lang-python,python,highlighted},language={python}]{\mdToken{Delimiter,Parenthesis,Python,BracketOpen}{(}\mdToken{Identifier,Python}{a}\mdToken{Operator,Python}{{\textless}}\mdToken{Identifier,Python}{c}\mdToken{Delimiter,Parenthesis,Python,BracketClose}{)}}% data-line={199}
{} only executes in the case where % data-line={199}
{}\mdCode[class={code,code1,language-python,lang-python,python,highlighted},language={python}]{\mdToken{Delimiter,Parenthesis,Python,BracketOpen}{(}\mdToken{Keyword,Python}{not}{\prespace{1}}\mdToken{Delimiter,Parenthesis,Python,BracketOpen}{(}\mdToken{Identifier,Python}{a}\mdToken{Operator,Python}{{\textless}}\mdToken{Identifier,Python}{b}\mdToken{Delimiter,Parenthesis,Python,BracketClose}{)}{\prespace{1}}\mdToken{Keyword,Python}{and}{\prespace{1}}\mdToken{Keyword,Python}{not}{\prespace{1}}\mdToken{Delimiter,Parenthesis,Python,BracketOpen}{(}\mdToken{Identifier,Python}{c}\mdToken{Operator,Python}{{\textless}}\mdToken{Identifier,Python}{d}\mdToken{Delimiter,Parenthesis,Python,BracketClose}{)}\mdToken{Delimiter,Parenthesis,Python,BracketClose}{)}}% data-line={199}
{}
holds.%
\end{mdP}%
\begin{mdP}[class={para-continue},data-line={208}]%
% data-line={208}
{}Integer solutions to the above three systems of constraints are:%
\end{mdP}%
\begin{mdUl}[class={list-star,compact},data-line={210}]%
\begin{mdLi}[data-line={210}]%
% data-line={210}
{}\mdCode[class={code,code1,language-python,lang-python,python,highlighted},language={python}]{\mdToken{Identifier,Python}{a}{\prespace{1}}\mdToken{Operator,Python}{==}{\prespace{1}}\mdToken{Number,Python}{0}{\prespace{1}}\mdToken{Keyword,Python}{and}{\prespace{1}}\mdToken{Identifier,Python}{b}{\prespace{1}}\mdToken{Operator,Python}{==}{\prespace{1}}\mdToken{Number,Python}{2}{\prespace{1}}\mdToken{Keyword,Python}{and}{\prespace{1}}\mdToken{Identifier,Python}{c}{\prespace{1}}\mdToken{Operator,Python}{==}{\prespace{1}}\mdToken{Number,Python}{0}{\prespace{1}}\mdToken{Keyword,Python}{and}{\prespace{1}}\mdToken{Identifier,Python}{d}{\prespace{1}}\mdToken{Operator,Python}{==}{\prespace{1}}\mdToken{Number,Python}{0}}% data-line={210}
{}%
\end{mdLi}%
\begin{mdLi}[data-line={211}]%
% data-line={211}
{}\mdCode[class={code,code1,language-python,lang-python,python,highlighted},language={python}]{\mdToken{Identifier,Python}{a}{\prespace{1}}\mdToken{Operator,Python}{==}{\prespace{1}}\mdToken{Number,Python}{0}{\prespace{1}}\mdToken{Keyword,Python}{and}{\prespace{1}}\mdToken{Identifier,Python}{b}{\prespace{1}}\mdToken{Operator,Python}{==}{\prespace{1}}\mdToken{Number,Python}{0}{\prespace{1}}\mdToken{Keyword,Python}{and}{\prespace{1}}\mdToken{Identifier,Python}{c}{\prespace{1}}\mdToken{Operator,Python}{==}{\prespace{1}}\mdToken{Number,Python}{0}{\prespace{1}}\mdToken{Keyword,Python}{and}{\prespace{1}}\mdToken{Identifier,Python}{d}{\prespace{1}}\mdToken{Operator,Python}{==}{\prespace{1}}\mdToken{Number,Python}{3}}% data-line={211}
{}%
\end{mdLi}%
\begin{mdLi}[data-line={212}]%
% data-line={212}
{}\mdCode[class={code,code1,language-python,lang-python,python,highlighted},language={python}]{\mdToken{Identifier,Python}{a}{\prespace{1}}\mdToken{Operator,Python}{==}{\prespace{1}}\mdToken{Number,Python}{0}{\prespace{1}}\mdToken{Keyword,Python}{and}{\prespace{1}}\mdToken{Identifier,Python}{b}{\prespace{1}}\mdToken{Operator,Python}{==}{\prespace{1}}\mdToken{Number,Python}{0}{\prespace{1}}\mdToken{Keyword,Python}{and}{\prespace{1}}\mdToken{Identifier,Python}{c}{\prespace{1}}\mdToken{Operator,Python}{==}{\prespace{1}}\mdToken{Number,Python}{2}{\prespace{1}}\mdToken{Keyword,Python}{and}{\prespace{1}}\mdToken{Identifier,Python}{d}{\prespace{1}}\mdToken{Operator,Python}{==}{\prespace{1}}\mdToken{Number,Python}{0}}% data-line={212}
{}%
\end{mdLi}%%
\end{mdUl}%
\begin{mdP}[data-line={214}]%
% data-line={214}
{}In the three cases above, we sought solutions that kept as many of
the variables as possible equal to the original input (in which 
all variables are equal to 0). Execution the % data-line={216}
{}\mdCode[class={code,code1,language-python,lang-python,python,highlighted},language={python}]{\mdToken{Identifier,Python}{max4}}% data-line={216}
{} function on the input
corresponding to the first solution produces the path-condition
% data-line={218}
{}\mdCode[class={code,code1,language-python,lang-python,python,highlighted},language={python}]{\mdToken{Delimiter,Parenthesis,Python,BracketOpen}{(}\mdToken{Delimiter,Parenthesis,Python,BracketOpen}{(}\mdToken{Identifier,Python}{a}\mdToken{Operator,Python}{{\textless}}\mdToken{Identifier,Python}{b}\mdToken{Delimiter,Parenthesis,Python,BracketClose}{)}{\prespace{1}}\mdToken{Keyword,Python}{and}{\prespace{1}}\mdToken{Keyword,Python}{not}\mdToken{Delimiter,Parenthesis,Python,BracketOpen}{(}\mdToken{Identifier,Python}{c}\mdToken{Operator,Python}{{\textless}}\mdToken{Identifier,Python}{d}\mdToken{Delimiter,Parenthesis,Python,BracketClose}{)}{\prespace{1}}\mdToken{Keyword,Python}{and}{\prespace{1}}\mdToken{Keyword,Python}{not}\mdToken{Delimiter,Parenthesis,Python,BracketOpen}{(}\mdToken{Identifier,Python}{b}{\prespace{1}}\mdToken{Operator,Python}{{\textless}}{\prespace{1}}\mdToken{Identifier,Python}{c}\mdToken{Delimiter,Parenthesis,Python,BracketClose}{)}\mdToken{Delimiter,Parenthesis,Python,BracketClose}{)}}% data-line={218}
{}, from which we can produce more
inputs.   For this (loop-free function), there are a finite number
of path-conditions. We leave it as an exercise to the reader to 
enumerate them all.%
\end{mdP}%
\mdHxxx[id=sec-leveraging-concrete-values-in-dse,label={[1.2]\{.heading-label\}},toc={},data-line={223},caption={[[1.2]\{.heading-label\}.{\enspace}]\{.heading-before\}Leveraging concrete values in DSE},bookmark={1.2.{\enspace}Leveraging concrete values in DSE}]{% data-line={223}
{}\mdSpan[class={heading-before}]{\mdSpan[class={heading-label}]{1.2}.{\enspace}}% data-line={223}
{}Leveraging concrete values in DSE}\begin{mdP}[data-line={225}]%
% data-line={225}
{}We now consider several situations where we can make use of concrete
values in DSE. In the realm of (unbounded-precision) integer arithmetic 
(e.g., bignum integer arithmetic, as in Python 3.0 onwards),
it is easy  to come up with  tiny programs that will be % data-line={228}
{}\mdEm{very difficult}% data-line={228}
{},
if not % data-line={229}
{}\mdEm{impossible}% data-line={229}
{}, 
for any symbolic execution tool to deal with, such as the function % data-line={230}
{}\mdCode[class={code,code1,language-python,lang-python,python,highlighted},language={python}]{\mdToken{Identifier,Python}{fermat3}}% data-line={230}
{}
in Figure% data-line={231}
{}{\mdNbsp}\mdA[class={localref},target-element={figure}]{fig-fermat3}{}{\mdSpan[class={figure-label}]{3}}% data-line={231}
{}.% data-line={231}
{} % data-line={231}
{}%
\end{mdP}%
\begin{mdDiv}[class={figure,align-center},id=fig-fermat3,label={[3]\{.figure-label\}},elem={figure},toc={tof},toc-line={[3]\{.figure-label\}. Hard example for symbolic execution},caption={Hard example for symbolic execution},data-line={234}]%
\begin{mdPre}[class={para-block,pre-fenced,pre-fenced3,language-python,lang-python,python,highlighted},language={python},data-line={235},data-line-code={236}]%
\mdPrecode[data-line={236}]{\mdToken{Keyword,Python}{def}{\prespace{1}}\mdToken{Identifier,Python}{fermat3}\mdToken{Delimiter,Parenthesis,Python,BracketOpen}{(}\mdToken{Identifier,Python}{x}\mdToken{Delimiter,Python}{,}\mdToken{Identifier,Python}{y}\mdToken{Delimiter,Python}{,}\mdToken{Identifier,Python}{z}\mdToken{Delimiter,Parenthesis,Python,BracketClose}{)}\mdToken{Keyword,Python,BracketOpen}{:}\prebr{}
{\preindent{3}}\mdToken{Keyword,Python}{if}{\prespace{1}}\mdToken{Delimiter,Parenthesis,Python,BracketOpen}{(}\mdToken{Identifier,Python}{x}{\prespace{1}}\mdToken{Operator,Python}{{\textgreater}}{\prespace{1}}\mdToken{Number,Python}{0}{\prespace{1}}\mdToken{Keyword,Python}{and}{\prespace{1}}\mdToken{Identifier,Python}{y}{\prespace{1}}\mdToken{Operator,Python}{{\textgreater}}{\prespace{1}}\mdToken{Number,Python}{0}{\prespace{1}}\mdToken{Keyword,Python}{and}{\prespace{1}}\mdToken{Identifier,Python}{z}{\prespace{1}}\mdToken{Operator,Python}{{\textgreater}}{\prespace{1}}\mdToken{Number,Python}{0}\mdToken{Delimiter,Parenthesis,Python,BracketClose}{)}\mdToken{Keyword,Python,BracketOpen}{:}\prebr{}
{\preindent{6}}\mdToken{Keyword,Python}{if}{\prespace{1}}\mdToken{Delimiter,Parenthesis,Python,BracketOpen}{(}\mdToken{Identifier,Python}{x}\mdToken{Operator,Python}{*}\mdToken{Identifier,Python}{x}\mdToken{Operator,Python}{*}\mdToken{Identifier,Python}{x}{\prespace{1}}\mdToken{Operator,Python}{+}{\prespace{1}}\mdToken{Identifier,Python}{y}\mdToken{Operator,Python}{*}\mdToken{Identifier,Python}{y}\mdToken{Operator,Python}{*}\mdToken{Identifier,Python}{y}{\prespace{1}}\mdToken{Operator,Python}{==}{\prespace{1}}\mdToken{Identifier,Python}{z}\mdToken{Operator,Python}{*}\mdToken{Identifier,Python}{z}\mdToken{Operator,Python}{*}\mdToken{Identifier,Python}{z}\mdToken{Delimiter,Parenthesis,Python,BracketClose}{)}\mdToken{Keyword,Python,BracketOpen}{:}\prebr{}
{\preindent{10}}\mdToken{Keyword,Python}{return}{\prespace{1}}\mdToken{String,Delim,Python,BracketOpen}{{"}}\mdToken{String,Python}{Fermat\prespace{1}and\prespace{1}Wiles\prespace{1}were\prespace{1}wrong!?!}\mdToken{String,Delim,Python,BracketClose}{{"}}\prebr{}
{\preindent{3}}\mdToken{Keyword,Python}{return}{\prespace{1}}\mdToken{Number,Python}{0}}%
\end{mdPre}%
\mdHr[class={figureline,madoko},data-line={242}]{}\begin{mdDiv}[data-line={243}]%
% data-line={243}
{}\mdSpan[class={figure-caption}]{\mdSpan[class={caption-before}]{\mdStrong{Figure{\mdNbsp}\mdSpan[class={figure-label}]{3}.} }Hard example for symbolic execution}% data-line={243}
{}%
\end{mdDiv}%%
\end{mdDiv}%
\begin{mdP}[class={indent},data-line={245}]%
% data-line={245}
{}Fermat% data-line={245}
{}{'}% data-line={245}
{}s Last Theorem, proved
by Andrew Wiles in the late 20th century, states that no 
three positive integers % data-line={247}
{}\mdSpan[class={math-inline},elem={math-inline}]{$x$}% data-line={247}
{}, % data-line={247}
{}\mdSpan[class={math-inline},elem={math-inline}]{$y$}% data-line={247}
{}, and % data-line={247}
{}\mdSpan[class={math-inline},elem={math-inline}]{$z$}% data-line={247}
{} can satisfy the equation
% data-line={248}
{}\mdSpan[class={math-inline},elem={math-inline}]{$x^n + y^n = z^n$}% data-line={248}
{} for any integer value of % data-line={248}
{}\mdSpan[class={math-inline},elem={math-inline}]{$n$}% data-line={248}
{} greater than two.
The function % data-line={249}
{}\mdCode[class={code,code1,language-python,lang-python,python,highlighted},language={python}]{\mdToken{Identifier,Python}{fermat3}}% data-line={249}
{} encodes this statement for % data-line={249}
{}\mdSpan[class={math-inline},elem={math-inline}]{$n=3$}% data-line={249}
{}.   It
is not reasonable to have a computer waste time trying to find
a solution that would cause % data-line={251}
{}\mdCode[class={code,code1,language-python,lang-python,python,highlighted},language={python}]{\mdToken{Identifier,Python}{fermat3}}% data-line={251}
{} to print the string 
% data-line={252}
{}\mdCode[class={code,code1,language-python,lang-python,python,highlighted},language={python}]{\mdToken{String,Delim,Python,BracketOpen}{{"}}\mdToken{String,Python}{Fermat\prespace{1}and\prespace{1}Wiles\prespace{1}were\prespace{1}wrong!?!}\mdToken{String,Delim,Python,BracketClose}{{"}}}% data-line={252}
{}.  In cases of complex (non-linear)
arithmetic operations,
such as % data-line={254}
{}\mdCode[class={code,code1,language-python,lang-python,python,highlighted},language={python}]{\mdToken{Identifier,Python}{x}\mdToken{Operator,Python}{*}\mdToken{Identifier,Python}{x}\mdToken{Operator,Python}{*}\mdToken{Identifier,Python}{x}}% data-line={254}
{}, we might choose to handle the operation concretely.%
\end{mdP}%
\begin{mdP}[class={indent},data-line={256}]%
% data-line={256}
{}There are a number of ways to deal with the above issue: one is 
to recognize all non-linear terms in a symbolic expression and replace
them with their concrete counterparts during execution. For the % data-line={258}
{}\mdCode[class={code,code1,language-python,lang-python,python,highlighted},language={python}]{\mdToken{Identifier,Python}{fermat3}}% data-line={258}
{} 
example, this would mean that during DSE the symbolic expression 
% data-line={260}
{}\mdCode[class={code,code1,language-python,lang-python,python,highlighted},language={python}]{\mdToken{Delimiter,Parenthesis,Python,BracketOpen}{(}\mdToken{Identifier,Python}{x}\mdToken{Operator,Python}{*}\mdToken{Identifier,Python}{x}\mdToken{Operator,Python}{*}\mdToken{Identifier,Python}{x}{\prespace{1}}\mdToken{Operator,Python}{+}{\prespace{1}}\mdToken{Identifier,Python}{y}\mdToken{Operator,Python}{*}\mdToken{Identifier,Python}{y}\mdToken{Operator,Python}{*}\mdToken{Identifier,Python}{y}{\prespace{1}}\mdToken{Operator,Python}{==}{\prespace{1}}\mdToken{Identifier,Python}{z}\mdToken{Operator,Python}{*}\mdToken{Identifier,Python}{z}\mdToken{Operator,Python}{*}\mdToken{Identifier,Python}{z}\mdToken{Delimiter,Parenthesis,Python,BracketClose}{)}}% data-line={260}
{} would be reduced to the constant % data-line={260}
{}\mdCode[class={code,code1,language-python,lang-python,python,highlighted},language={python}]{\mdToken{Namespace,Identifier,Python}{False}}% data-line={260}
{}
by evaluation on the concrete values of variables % data-line={261}
{}\mdCode[class={code,code1,language-python,lang-python,python,highlighted},language={python}]{\mdToken{Identifier,Python}{x}}% data-line={261}
{}, % data-line={261}
{}\mdCode[class={code,code1,language-python,lang-python,python,highlighted},language={python}]{\mdToken{Identifier,Python}{y}}% data-line={261}
{} and % data-line={261}
{}\mdCode[class={code,code1,language-python,lang-python,python,highlighted},language={python}]{\mdToken{Identifier,Python}{z}}% data-line={261}
{}.%
\end{mdP}%
\begin{mdP}[class={indent},data-line={263}]%
% data-line={263}
{}Besides difficult operations (such as non-linear arithmetic),
other examples of code that we might treat
concretely instead of symbolically include
functions that are hard to invert, such as cryptographic hash functions,
or low-level functions that we do not wish to test (such as 
operating system functions).  Consider
the code in Figure% data-line={269}
{}{\mdNbsp}\mdA[class={localref},target-element={figure}]{fig-hash}{}{\mdSpan[class={figure-label}]{4}}% data-line={269}
{}, which applies the function
% data-line={270}
{}\mdCode[class={code,code1,language-python,lang-python,python,highlighted},language={python}]{\mdToken{Identifier,Python}{unknown}}% data-line={270}
{} to argument % data-line={270}
{}\mdCode[class={code,code1,language-python,lang-python,python,highlighted},language={python}]{\mdToken{Identifier,Python}{x}}% data-line={270}
{} and compares it to argument % data-line={270}
{}\mdCode[class={code,code1,language-python,lang-python,python,highlighted},language={python}]{\mdToken{Identifier,Python}{y}}% data-line={270}
{}.
By using the name % data-line={271}
{}\mdCode[class={code,code1,language-python,lang-python,python,highlighted},language={python}]{\mdToken{Identifier,Python}{unknown}}% data-line={271}
{} we simply mean to say that we 
wish to model this function as a black box, with no knowledge
of how it operates internally.%
\end{mdP}%
\begin{mdDiv}[class={figure,align-center},id=fig-hash,label={[4]\{.figure-label\}},elem={figure},toc={tof},toc-line={[4]\{.figure-label\}. Another hard example for symbolic execution},caption={Another hard example for symbolic execution},data-line={275}]%
\begin{mdPre}[class={para-block,pre-fenced,pre-fenced3,language-python,lang-python,python,highlighted},language={python},data-line={276},data-line-code={277}]%
\mdPrecode[data-line={277}]{\mdToken{Keyword,Python}{def}{\prespace{1}}\mdToken{Identifier,Python}{dart}\mdToken{Delimiter,Parenthesis,Python,BracketOpen}{(}\mdToken{Identifier,Python}{x}\mdToken{Delimiter,Python}{,}\mdToken{Identifier,Python}{y}\mdToken{Delimiter,Parenthesis,Python,BracketClose}{)}\mdToken{Keyword,Python,BracketOpen}{:}\prebr{}
{\preindent{2}}\mdToken{Keyword,Python}{if}{\prespace{1}}\mdToken{Delimiter,Parenthesis,Python,BracketOpen}{(}\mdToken{Identifier,Python}{unknown}\mdToken{Delimiter,Parenthesis,Python,BracketOpen}{(}\mdToken{Identifier,Python}{x}\mdToken{Delimiter,Parenthesis,Python,BracketClose}{)}{\prespace{1}}\mdToken{Operator,Python}{==}{\prespace{1}}\mdToken{Identifier,Python}{y}\mdToken{Delimiter,Parenthesis,Python,BracketClose}{)}\mdToken{Keyword,Python,BracketOpen}{:}\prebr{}
{\preindent{5}}\mdToken{Keyword,Python}{return}{\prespace{1}}\mdToken{Number,Python}{1}\prebr{}
{\preindent{2}}\mdToken{Keyword,Python}{return}{\prespace{1}}\mdToken{Number,Python}{0}}%
\end{mdPre}%
\mdHr[class={figureline,madoko},data-line={282}]{}\begin{mdDiv}[data-line={283}]%
% data-line={283}
{}\mdSpan[class={figure-caption}]{\mdSpan[class={caption-before}]{\mdStrong{Figure{\mdNbsp}\mdSpan[class={figure-label}]{4}.} }Another hard example for symbolic execution}% data-line={283}
{}%
\end{mdDiv}%%
\end{mdDiv}%
\begin{mdP}[class={indent},data-line={284}]%
% data-line={284}
{}In such a case, we can use DSE to execute the function % data-line={284}
{}\mdCode[class={code,code1,language-python,lang-python,python,highlighted},language={python}]{\mdToken{Identifier,Python}{unknown}}% data-line={284}
{}
on a specific input (say % data-line={285}
{}\mdCode[class={code,code1,language-python,lang-python,python,highlighted},language={python}]{\mdToken{Number,Python}{5013}}% data-line={285}
{}) and observe its output
(say % data-line={286}
{}\mdCode[class={code,code1,language-python,lang-python,python,highlighted},language={python}]{\mdToken{Number,Python}{42}}% data-line={286}
{}). That is, rather than execute % data-line={286}
{}\mdCode[class={code,code1,language-python,lang-python,python,highlighted},language={python}]{\mdToken{Identifier,Python}{unknown}}% data-line={286}
{} symbolically
and invoke an ATP to invert the function% data-line={287}
{}{'}% data-line={287}
{}s path-condition, we
simply treat the call to % data-line={288}
{}\mdCode[class={code,code1,language-python,lang-python,python,highlighted},language={python}]{\mdToken{Identifier,Python}{unknown}}% data-line={288}
{} concretely, substituting
its return value (in this case % data-line={289}
{}\mdCode[class={code,code1,language-python,lang-python,python,highlighted},language={python}]{\mdToken{Number,Python}{42}}% data-line={289}
{}) for the specialized expression 
% data-line={290}
{}\mdCode[class={code,code1,language-python,lang-python,python,highlighted},language={python}]{\mdToken{Identifier,Python}{unknown}\mdToken{Delimiter,Parenthesis,Python,BracketOpen}{(}\mdToken{Number,Python}{5013}\mdToken{Delimiter,Parenthesis,Python,BracketClose}{)}{\prespace{1}}\mdToken{Operator,Python}{==}{\prespace{1}}\mdToken{Identifier,Python}{y}}% data-line={290}
{} to get the predicate % data-line={290}
{}\mdCode[class={code,code1,language-python,lang-python,python,highlighted},language={python}]{\mdToken{Delimiter,Parenthesis,Python,BracketOpen}{(}\mdToken{Number,Python}{42}{\prespace{1}}\mdToken{Operator,Python}{==}{\prespace{1}}\mdToken{Identifier,Python}{y}\mdToken{Delimiter,Parenthesis,Python,BracketClose}{)}}% data-line={290}
{}.%
\end{mdP}%
\begin{mdP}[class={indent},data-line={292}]%
% data-line={292}
{}Adding the constraint % data-line={292}
{}\mdCode[class={code,code1,language-python,lang-python,python,highlighted},language={python}]{\mdToken{Delimiter,Parenthesis,Python,BracketOpen}{(}\mdToken{Identifier,Python}{x}{\prespace{1}}\mdToken{Operator,Python}{==}{\prespace{1}}\mdToken{Number,Python}{5013}\mdToken{Delimiter,Parenthesis,Python,BracketClose}{)}}% data-line={292}
{} yields the sound but
rather specific path-condition 
% data-line={294}
{}\mdCode[class={code,code1,language-python,lang-python,python,highlighted},language={python}]{\mdToken{Delimiter,Parenthesis,Python,BracketOpen}{(}\mdToken{Identifier,Python}{x}{\prespace{1}}\mdToken{Operator,Python}{==}{\prespace{1}}\mdToken{Number,Python}{5013}\mdToken{Delimiter,Parenthesis,Python,BracketClose}{)}{\prespace{1}}\mdToken{Keyword,Python}{and}{\prespace{1}}\mdToken{Delimiter,Parenthesis,Python,BracketOpen}{(}\mdToken{Number,Python}{42}{\prespace{1}}\mdToken{Operator,Python}{==}{\prespace{1}}\mdToken{Identifier,Python}{y}\mdToken{Delimiter,Parenthesis,Python,BracketClose}{)}}% data-line={294}
{}.
Note that the path-condition % data-line={295}
{}\mdCode[class={code,code1,language-python,lang-python,python,highlighted},language={python}]{\mdToken{Delimiter,Parenthesis,Python,BracketOpen}{(}\mdToken{Number,Python}{42}{\prespace{1}}\mdToken{Operator,Python}{==}{\prespace{1}}\mdToken{Identifier,Python}{y}\mdToken{Delimiter,Parenthesis,Python,BracketClose}{)}}% data-line={295}
{} is not sound, as it admits
any value for the variable % data-line={296}
{}\mdCode[class={code,code1,language-python,lang-python,python,highlighted},language={python}]{\mdToken{Identifier,Python}{x}}% data-line={296}
{}, which likely includes many values
for which % data-line={297}
{}\mdCode[class={code,code1,language-python,lang-python,python,highlighted},language={python}]{\mdToken{Delimiter,Parenthesis,Python,BracketOpen}{(}\mdToken{Identifier,Python}{unknown}\mdToken{Delimiter,Parenthesis,Python,BracketOpen}{(}\mdToken{Identifier,Python}{x}\mdToken{Delimiter,Parenthesis,Python,BracketClose}{)}{\prespace{1}}\mdToken{Operator,Python}{==}{\prespace{1}}\mdToken{Identifier,Python}{y}\mdToken{Delimiter,Parenthesis,Python,BracketClose}{)}}% data-line={297}
{} is false.%
\end{mdP}%
\mdHxxx[id=sec-overview,label={[1.3]\{.heading-label\}},toc={},data-line={299},caption={[[1.3]\{.heading-label\}.{\enspace}]\{.heading-before\}Overview},bookmark={1.3.{\enspace}Overview}]{% data-line={299}
{}\mdSpan[class={heading-before}]{\mdSpan[class={heading-label}]{1.3}.{\enspace}}% data-line={299}
{}Overview}\begin{mdP}[class={para-continue},data-line={301}]%
% data-line={301}
{}This introduction elides many important 
issues that arise in implementing DSE for a real language, which we will 
focus on in the remainder of the paper. These include how to:%
\end{mdP}%
\begin{mdUl}[class={list-star,compact},data-line={305}]%
\begin{mdLi}[data-line={305}]%
% data-line={305}
{}Identify the code under test % data-line={305}
{}\mdSpan[class={math-inline},elem={math-inline}]{$P$}% data-line={305}
{} and the symbolic inputs to % data-line={305}
{}\mdSpan[class={math-inline},elem={math-inline}]{$P$}% data-line={305}
{};%
\end{mdLi}%
\begin{mdLi}[data-line={306}]%
% data-line={306}
{}Trace the control flow path % data-line={306}
{}\mdSpan[class={math-inline},elem={math-inline}]{$p$}% data-line={306}
{} taken by execution % data-line={306}
{}\mdSpan[class={math-inline},elem={math-inline}]{$P(i)$}% data-line={306}
{};%
\end{mdLi}%
\begin{mdLi}[data-line={307}]%
% data-line={307}
{}Reinterpret program operations to compute symbolic expressions;%
\end{mdLi}%
\begin{mdLi}[data-line={308}]%
% data-line={308}
{}Generate a path-condition from % data-line={308}
{}\mdSpan[class={math-inline},elem={math-inline}]{$p$}% data-line={308}
{} and the symbolic expressions;%
\end{mdLi}%
\begin{mdLi}[data-line={309}]%
% data-line={309}
{}Generate a new input % data-line={309}
{}\mdSpan[class={math-inline},elem={math-inline}]{$i'$}% data-line={309}
{} by negating (part of) the path-condition, translating
the path-condition to the input language of an ATP, invoking the ATP, and
lifting a satisfying model (if any) back up to the source level;%
\end{mdLi}%
\begin{mdLi}[data-line={312}]%
% data-line={312}
{}Guide the search to expose new paths.%
\end{mdLi}%%
\end{mdUl}%
\begin{mdP}[data-line={314}]%
% data-line={314}
{}The rest of this paper is organized as follows. Section% data-line={314}
{}{\mdNbsp}\mdA[class={localref},target-element={h1}]{sec-semantics}{}{\mdSpan[class={heading-label}]{2}}% data-line={314}
{} 
describes an instrumented typing discipline where we lift each type (representing 
a set of concrete values) to a symbolic type (representing
a set of pairs of concrete and symbolic values).
Section% data-line={318}
{}{\mdNbsp}\mdA[class={localref},target-element={h1}]{sec-sp2dse}{}{\mdSpan[class={heading-label}]{3}}% data-line={318}
{} shows how strongest postconditions defines a symbolic
semantics for a small programming language and how strongest postconditions
can be refined to model DSE.
Section% data-line={321}
{}{\mdNbsp}\mdA[class={localref},target-element={h1}]{sec-impl}{}{\mdSpan[class={heading-label}]{4}}% data-line={321}
{} describes an implementation of DSE for the Python language
in the Python language that follows the instrumented semantics pattern closely
(full implementation and tests available at% data-line={323}
{}{\mdNbsp}\mdA{https://github.com/thomasjball/PyExZ3/}{}{PyExZ3}% data-line={323}
{}, tagged % data-line={323}
{}{\textquotedblleft}v1.0{\textquotedblright}% data-line={323}
{}).
Section% data-line={324}
{}{\mdNbsp}\mdA[class={localref},target-element={h1}]{sec-int2z3}{}{\mdSpan[class={heading-label}]{5}}% data-line={324}
{} describes the symbolic encoding of Python integer 
operations using two decision procedures of Z3: linear arithmetic with
uninterpreted functions in place of non-linear operations;
fixed-width bitvectors with precise encodings of most operations.
Section% data-line={328}
{}{\mdNbsp}\mdA[class={localref},target-element={h1}]{sec-extensions}{}{\mdSpan[class={heading-label}]{6}}% data-line={328}
{} offers a number of ideas for projects
to extend the capabilities of% data-line={329}
{}{\mdNbsp}\mdA{https://github.com/thomasjball/PyExZ3/}{}{PyExZ3}% data-line={329}
{}.%
\end{mdP}%
\mdHxx[id=sec-semantics,label={[2]\{.heading-label\}},toc={},data-line={331},caption={[[2]\{.heading-label\}.{\enspace}]\{.heading-before\}Instrumented Types},bookmark={2.{\enspace}Instrumented Types}]{% data-line={331}
{}\mdSpan[class={heading-before}]{\mdSpan[class={heading-label}]{2}.{\enspace}}% data-line={331}
{}Instrumented Types}\begin{mdP}[data-line={333}]%
% data-line={333}
{}We are given a universe of classes/types % data-line={333}
{}\mdSpan[class={math-inline},elem={math-inline}]{$U$}% data-line={333}
{}; a type % data-line={333}
{}\mdSpan[class={math-inline},elem={math-inline}]{$T \in U$}% data-line={333}
{} carries
along a  set of operations that apply to values of type % data-line={334}
{}\mdSpan[class={math-inline},elem={math-inline}]{$T$}% data-line={334}
{},
where an operation % data-line={335}
{}\mdSpan[class={math-inline},elem={math-inline}]{$o \in T$}% data-line={335}
{}  takes an argument list of typed 
values as input (the first being of type % data-line={336}
{}\mdSpan[class={math-inline},elem={math-inline}]{$T$}% data-line={336}
{}) and produces a single 
typed value as output. Nullary (static) operations of type % data-line={337}
{}\mdSpan[class={math-inline},elem={math-inline}]{$T$}% data-line={337}
{} can be 
used to create values of type % data-line={338}
{}\mdSpan[class={math-inline},elem={math-inline}]{$T$}% data-line={338}
{} (such as constants, objects, etc.)%
\end{mdP}%
\begin{mdP}[class={indent,para-continue},data-line={340}]%
% data-line={340}
{}A program % data-line={340}
{}\mdSpan[class={math-inline},elem={math-inline}]{$P$}% data-line={340}
{} has typed input variables
% data-line={341}
{}\mdSpan[class={math-inline},elem={math-inline}]{$v_1 : T_1 \ldots v_k : T_k$}% data-line={341}
{} and a body from the language of statements % data-line={341}
{}\mdSpan[class={math-inline},elem={math-inline}]{$S$}% data-line={341}
{}:%
\end{mdP}%
\begin{mdDiv}[class={mathpre,para-block,input-mathpre},elem={mathpre},data-line={343}]%
\begin{mdDiv}[class={math-display}]%
\[% data-line={344}
\begin{mdMathprearray}%
\mathid{S}\mdMathspace{1}\rightarrow   &\mdMathspace{1}\mathid{v}\mdMathspace{1}:=\mdMathspace{1}\mathid{E}\mdMathbr{}
\mdMathindent{3}|\mdMathspace{1}&\mdMathspace{1}\mathkw{skip}\mdMathspace{1}\mdMathbr{}
\mdMathindent{3}|\mdMathspace{1}&\mdMathspace{1}\mathid{S}_1\mdMathspace{1};\mdMathspace{1}\mathid{S}_2\mdMathspace{1}\mdMathbr{}
\mdMathindent{3}|\mdMathspace{1}&\mdMathspace{1}\mathkw{if}\mdMathspace{1}\mathid{E}\mdMathspace{1}\mathkw{then}\mdMathspace{1}\mathid{S}_1\mdMathspace{1}\mathkw{else}\mdMathspace{1}\mathid{S}_2\mdMathspace{1}\mathkw{end}\mdMathbr{}
\mdMathindent{3}|\mdMathspace{1}&\mdMathspace{1}\mathkw{while}\mdMathspace{1}\mathid{E}\mdMathspace{1}\mathkw{do}\mdMathspace{1}\mathid{S}\mdMathspace{1}\mathkw{end}
\end{mdMathprearray}\]%
\end{mdDiv}%%
\end{mdDiv}%
\begin{mdP}[data-line={351}]%
% data-line={351}
{}The language of expressions (% data-line={351}
{}\mdSpan[class={math-inline},elem={math-inline}]{$E$}% data-line={351}
{}) is defined by the application of operations
to values, where constants (nullary operations) and 
program variables form the leaves 
of the expression tree and non-nullary operators 
form the interior nodes of the tree.
For now, we will consider all values to be immutable.
That is, the only source of mutation in the language is the 
assignment statement.%
\end{mdP}%
\begin{mdP}[class={indent,para-continue},data-line={360}]%
% data-line={360}
{}To introduce symbolic execution into the picture,
we can imagine that a type % data-line={361}
{}\mdSpan[class={math-inline},elem={math-inline}]{$T \in U$}% data-line={361}
{} has
(one or more) counterparts in a symbolic universe % data-line={362}
{}\mdSpan[class={math-inline},elem={math-inline}]{$U'$}% data-line={362}
{}. A type % data-line={362}
{}\mdSpan[class={math-inline},elem={math-inline}]{$T' \in U'$}% data-line={362}
{}
is a subtype of % data-line={363}
{}\mdSpan[class={math-inline},elem={math-inline}]{$T \in U$}% data-line={363}
{} with two purposes:%
\end{mdP}%
\begin{mdUl}[class={list-star,loose},data-line={365}]%
\begin{mdLi}[data-line={365}]%
\begin{mdP}[data-line={365}]%
% data-line={365}
{}First, a value of type % data-line={365}
{}\mdSpan[class={math-inline},elem={math-inline}]{$T'$}% data-line={365}
{} represents a pair of values: 
a concrete value % data-line={366}
{}\mdSpan[class={math-inline},elem={math-inline}]{$c$}% data-line={366}
{} of (super)type % data-line={366}
{}\mdSpan[class={math-inline},elem={math-inline}]{$T$}% data-line={366}
{} and a symbolic expression % data-line={366}
{}\mdSpan[class={math-inline},elem={math-inline}]{$e$}% data-line={366}
{}. 
A symbolic expression is a tree
whose leaves are either nullary operators (i.e., constants) of a type in % data-line={368}
{}\mdSpan[class={math-inline},elem={math-inline}]{$U$}% data-line={368}
{}
or are Skolem constants representing the (symbolic) inputs (% data-line={369}
{}\mdSpan[class={math-inline},elem={math-inline}]{$v_1 \ldots v_k$}% data-line={369}
{})
to the program % data-line={370}
{}\mdSpan[class={math-inline},elem={math-inline}]{$P$}% data-line={370}
{}, and whose interior nodes represent operations 
from types in % data-line={371}
{}\mdSpan[class={math-inline},elem={math-inline}]{$U$}% data-line={371}
{}. We refer to Skolem constants as % data-line={371}
{}{\textquotedblleft}symbolic constants{\textquotedblright}% data-line={371}
{}
from this point on. Note that symbolic expressions do not contain
references to program variables.%
\end{mdP}%%
\end{mdLi}%
\begin{mdLi}[data-line={375}]%
\begin{mdP}[data-line={375}]%
% data-line={375}
{}Second, the type % data-line={375}
{}\mdSpan[class={math-inline},elem={math-inline}]{$T'$}% data-line={375}
{} redefines some of the operations % data-line={375}
{}\mdSpan[class={math-inline},elem={math-inline}]{$o \in T$}% data-line={375}
{},
namely those for which we wish to compute symbolic expressions.
An operation % data-line={377}
{}\mdSpan[class={math-inline},elem={math-inline}]{$o \in T'$}% data-line={377}
{} has the same parameter list as % data-line={377}
{}\mdSpan[class={math-inline},elem={math-inline}]{$o \in T$}% data-line={377}
{}, allowing it
to take inputs with types from both % data-line={378}
{}\mdSpan[class={math-inline},elem={math-inline}]{$U$}% data-line={378}
{} and % data-line={378}
{}\mdSpan[class={math-inline},elem={math-inline}]{$U'$}% data-line={378}
{}. The return type
of % data-line={379}
{}\mdSpan[class={math-inline},elem={math-inline}]{$o \in T'$}% data-line={379}
{} generally is from % data-line={379}
{}\mdSpan[class={math-inline},elem={math-inline}]{$U'$}% data-line={379}
{} (though it can be from % data-line={379}
{}\mdSpan[class={math-inline},elem={math-inline}]{$U$}% data-line={379}
{}). 
Thus, % data-line={380}
{}\mdSpan[class={math-inline},elem={math-inline}]{$o \in T'$}% data-line={380}
{} is a proper function subtype of % data-line={380}
{}\mdSpan[class={math-inline},elem={math-inline}]{$o \in T$}% data-line={380}
{}. 
The purpose of % data-line={381}
{}\mdSpan[class={math-inline},elem={math-inline}]{$o \in T'$}% data-line={381}
{} is to:
(1) perform operation % data-line={382}
{}\mdSpan[class={math-inline},elem={math-inline}]{$o \in T$}% data-line={382}
{} on the concrete 
values associated with its inputs; 
(2) build a symbolic expression tree rooted at operation % data-line={384}
{}\mdSpan[class={math-inline},elem={math-inline}]{$o$}% data-line={384}
{} 
whose children are the trees associated with the inputs to % data-line={385}
{}\mdSpan[class={math-inline},elem={math-inline}]{$o$}% data-line={385}
{}.%
\end{mdP}%%
\end{mdLi}%%
\end{mdUl}%
\begin{mdP}[data-line={387}]%
% data-line={387}
{}Figure% data-line={387}
{}{\mdNbsp}\mdA[class={localref},target-element={figure}]{fig-subtype}{}{\mdSpan[class={figure-label}]{5}}% data-line={387}
{} presents pseudo code for the instrumentation
of a type % data-line={388}
{}\mdSpan[class={math-inline},elem={math-inline}]{$T$}% data-line={388}
{} via a type % data-line={388}
{}\mdSpan[class={math-inline},elem={math-inline}]{$T'$}% data-line={388}
{}.
The class % data-line={389}
{}\mdCode[class={code,code2,language-cpp2,lang-cpp2,cpp2,highlighted},language={cpp2}]{\mdToken{Type,Identifier,Cpp}{Symbolic}}% data-line={389}
{} is used to hold an expression tree (% data-line={389}
{}\mdCode[class={code,code2,language-cpp2,lang-cpp2,cpp2,highlighted},language={cpp2}]{\mdToken{Type,Identifier,Cpp}{Expr}}% data-line={389}
{}).
Given a class % data-line={390}
{}\mdSpan[class={math-inline},elem={math-inline}]{$T \in U$}% data-line={390}
{}, a symbolic type % data-line={390}
{}\mdSpan[class={math-inline},elem={math-inline}]{$T' \in U'$}% data-line={390}
{} is defined by inheriting 
from both % data-line={391}
{}\mdSpan[class={math-inline},elem={math-inline}]{$T$}% data-line={391}
{} and % data-line={391}
{}\mdCode[class={code,code2,language-cpp2,lang-cpp2,cpp2,highlighted},language={cpp2}]{\mdToken{Type,Identifier,Cpp}{Symbolic}}% data-line={391}
{}. This ensures that a % data-line={391}
{}\mdSpan[class={math-inline},elem={math-inline}]{$T'$}% data-line={391}
{} can be used
wherever a % data-line={392}
{}\mdSpan[class={math-inline},elem={math-inline}]{$T$}% data-line={392}
{} is expected.%
\end{mdP}%
\begin{mdDiv}[class={figure,align-center},id=fig-subtype,label={[5]\{.figure-label\}},elem={figure},toc={tof},toc-line={[5]\{.figure-label\}. Type instrumentation to carry both concrete values and symbolic expressions.},caption={Type instrumentation to carry both concrete values and symbolic expressions.},data-line={394}]%
\begin{mdPre}[class={para-block,pre-fenced,pre-fenced3,language-cpp2,lang-cpp2,cpp2,highlighted},data-line={395},data-line-code={396},language={cpp2}]%
\mdPrecode[data-line={396}]{{\preindent{2}}\mdToken{Keyword,Cpp}{class}{\prespace{1}}\mdToken{Type,Identifier,Cpp}{\mdSpan[class={code-escaped}]{\mdSpan[class={math-inline},elem={math-inline}]{$T'$}}}{\prespace{1}}\mdToken{Operator,Cpp}{:}{\prespace{1}}\mdToken{Type,Identifier,Cpp}{\mdSpan[class={code-escaped}]{\mdSpan[class={math-inline},elem={math-inline}]{$T$}}}\mdToken{Delimiter,Cpp}{,}{\prespace{1}}\mdToken{Type,Identifier,Cpp}{Symbolic}{\prespace{1}}\mdToken{Delimiter,Curly,Cpp,BracketOpen}{\{}\prebr{}
{\preindent{4}}\mdToken{Type,Identifier,Cpp}{\mdSpan[class={code-escaped}]{\mdSpan[class={math-inline},elem={math-inline}]{$T'$}}}\mdToken{Delimiter,Parenthesis,Cpp,BracketOpen}{(}\mdToken{Identifier,Cpp}{c}\mdToken{Operator,Cpp}{:}\mdToken{Type,Identifier,Cpp}{\mdSpan[class={code-escaped}]{\mdSpan[class={math-inline},elem={math-inline}]{$T$}}}\mdToken{Delimiter,Cpp}{,}{\prespace{1}}\mdToken{Identifier,Cpp}{e}\mdToken{Operator,Cpp}{:}\mdToken{Type,Identifier,Cpp}{Expr}\mdToken{Delimiter,Parenthesis,Cpp,BracketClose}{)}{\prespace{1}}\mdToken{Operator,Cpp}{:}{\prespace{1}}\mdToken{Type,Identifier,Cpp}{\mdSpan[class={code-escaped}]{\mdSpan[class={math-inline},elem={math-inline}]{$T$}}}\mdToken{Delimiter,Parenthesis,Cpp,BracketOpen}{(}\mdToken{Identifier,Cpp}{c}\mdToken{Delimiter,Parenthesis,Cpp,BracketClose}{)}\mdToken{Delimiter,Cpp}{,}{\prespace{1}}\mdToken{Source,Cpp}{S}\mdToken{Identifier,Cpp}{ymbolic}\mdToken{Delimiter,Parenthesis,Cpp,BracketOpen}{(}\mdToken{Identifier,Cpp}{e}\mdToken{Delimiter,Parenthesis,Cpp,BracketClose}{)}{\prespace{1}}\mdToken{Delimiter,Curly,Cpp,BracketOpen}{\{}\mdToken{Delimiter,Curly,Cpp,BracketClose}{\}}\prebr{}
\prebr{}
{\preindent{4}}\mdToken{Keyword,Extra,Cpp}{override}{\prespace{1}}\mdToken{Identifier,Cpp}{o}\mdToken{Delimiter,Parenthesis,Cpp,BracketOpen}{(}\mdToken{Keyword,Cpp}{this}\mdToken{Operator,Cpp}{:}\mdToken{Type,Identifier,Cpp}{\mdSpan[class={code-escaped}]{\mdSpan[class={math-inline},elem={math-inline}]{$T$}}}\mdToken{Delimiter,Cpp}{,}{\prespace{1}}\mdToken{Identifier,Cpp}{f1}\mdToken{Operator,Cpp}{:}\mdToken{Type,Identifier,Cpp}{\mdSpan[class={code-escaped}]{\mdSpan[class={math-inline},elem={math-inline}]{$T_1$}}}\mdToken{Delimiter,Cpp}{,}{\prespace{1}}\mdToken{Delimiter,Cpp}{.}\mdToken{Delimiter,Cpp}{.}\mdToken{Delimiter,Cpp}{.}{\prespace{1}}\mdToken{Delimiter,Cpp}{,}{\prespace{1}}\mdToken{Identifier,Cpp}{fk}\mdToken{Operator,Cpp}{:}\mdToken{Type,Identifier,Cpp}{\mdSpan[class={code-escaped}]{\mdSpan[class={math-inline},elem={math-inline}]{$T_k$}}}\mdToken{Delimiter,Parenthesis,Cpp,BracketClose}{)}{\prespace{1}}\mdToken{Operator,Cpp}{:}{\prespace{1}}\mdToken{Type,Identifier,Cpp}{\mdSpan[class={code-escaped}]{\mdSpan[class={math-inline},elem={math-inline}]{$R'$}}}{\prespace{1}}\mdToken{Delimiter,Curly,Cpp,BracketOpen}{\{}\prebr{}
{\preindent{6}}\mdToken{Keyword,Extra,Cpp}{var}{\prespace{1}}\mdToken{Identifier,Cpp}{c}{\prespace{1}}:={\prespace{1}}\mdToken{Type,Identifier,Cpp}{\mdSpan[class={code-escaped}]{\mdSpan[class={math-inline},elem={math-inline}]{$T$}}}\mdToken{Delimiter,Cpp}{.}\mdToken{Identifier,Cpp}{o}\mdToken{Delimiter,Parenthesis,Cpp,BracketOpen}{(}\mdToken{Keyword,Cpp}{this}\mdToken{Delimiter,Cpp}{,}{\prespace{1}}\mdToken{Identifier,Cpp}{f1}\mdToken{Delimiter,Cpp}{,}{\prespace{1}}\mdToken{Delimiter,Cpp}{.}\mdToken{Delimiter,Cpp}{.}\mdToken{Delimiter,Cpp}{.}{\prespace{1}}\mdToken{Delimiter,Cpp}{,}\mdToken{Identifier,Cpp}{fk}\mdToken{Delimiter,Parenthesis,Cpp,BracketClose}{)}\prebr{}
{\preindent{6}}\mdToken{Keyword,Extra,Cpp}{var}{\prespace{1}}\mdToken{Identifier,Cpp}{e}{\prespace{1}}:={\prespace{1}}\mdToken{Keyword,Cpp}{new}{\prespace{1}}\mdToken{Source,Cpp}{E}\mdToken{Identifier,Cpp}{xpr}\mdToken{Delimiter,Parenthesis,Cpp,BracketOpen}{(}\mdToken{Type,Identifier,Cpp}{\mdSpan[class={code-escaped}]{\mdSpan[class={math-inline},elem={math-inline}]{$T$}}}\mdToken{Delimiter,Cpp}{.}\mdToken{Identifier,Cpp}{o}\mdToken{Delimiter,Cpp}{,}{\prespace{1}}\mdToken{Identifier,Cpp}{expr}\mdToken{Delimiter,Parenthesis,Cpp,BracketOpen}{(}\mdToken{Identifier,Cpp}{self}\mdToken{Delimiter,Parenthesis,Cpp,BracketClose}{)}\mdToken{Delimiter,Cpp}{,}{\prespace{1}}\mdToken{Identifier,Cpp}{expr}\mdToken{Delimiter,Parenthesis,Cpp,BracketOpen}{(}\mdToken{Identifier,Cpp}{f1}\mdToken{Delimiter,Parenthesis,Cpp,BracketClose}{)}\mdToken{Delimiter,Cpp}{,}{\prespace{1}}\mdToken{Delimiter,Cpp}{.}\mdToken{Delimiter,Cpp}{.}\mdToken{Delimiter,Cpp}{.}\mdToken{Delimiter,Cpp}{,}{\prespace{1}}\mdToken{Identifier,Cpp}{expr}\mdToken{Delimiter,Parenthesis,Cpp,BracketOpen}{(}\mdToken{Identifier,Cpp}{fk}\mdToken{Delimiter,Parenthesis,Cpp,BracketClose}{)}\mdToken{Delimiter,Parenthesis,Cpp,BracketClose}{)}\prebr{}
{\preindent{6}}\mdToken{Keyword,Cpp}{return}{\prespace{1}}\mdToken{Keyword,Cpp}{new}{\prespace{1}}\mdToken{Type,Identifier,Cpp}{\mdSpan[class={code-escaped}]{\mdSpan[class={math-inline},elem={math-inline}]{$R'$}}}\mdToken{Delimiter,Parenthesis,Cpp,BracketOpen}{(}\mdToken{Identifier,Cpp}{c}\mdToken{Delimiter,Cpp}{,}\mdToken{Identifier,Cpp}{e}\mdToken{Delimiter,Parenthesis,Cpp,BracketClose}{)}{\prespace{1}}\prebr{}
{\preindent{4}}\mdToken{Delimiter,Curly,Cpp,BracketClose}{\}}\prebr{}
{\preindent{4}}\mdToken{Delimiter,Cpp}{.}\mdToken{Delimiter,Cpp}{.}\mdToken{Delimiter,Cpp}{.}\prebr{}
{\preindent{2}}\mdToken{Delimiter,Curly,Cpp,BracketClose}{\}}\prebr{}
{\preindent{2}}\prebr{}
{\preindent{2}}\mdToken{Keyword,Cpp}{class}{\prespace{1}}\mdToken{Type,Identifier,Cpp}{\mdSpan[class={code-escaped}]{\mdSpan[class={math-inline},elem={math-inline}]{$R'$}}}{\prespace{1}}\mdToken{Operator,Cpp}{:}{\prespace{1}}\mdToken{Type,Identifier,Cpp}{\mdSpan[class={code-escaped}]{\mdSpan[class={math-inline},elem={math-inline}]{$R$}}}\mdToken{Delimiter,Cpp}{,}{\prespace{1}}\mdToken{Type,Identifier,Cpp}{Symbolic}{\prespace{1}}\mdToken{Delimiter,Curly,Cpp,BracketOpen}{\{}{\prespace{1}}\mdToken{Delimiter,Cpp}{.}\mdToken{Delimiter,Cpp}{.}\mdToken{Delimiter,Cpp}{.}{\prespace{1}}\mdToken{Delimiter,Curly,Cpp,BracketClose}{\}}\prebr{}
{\preindent{2}}\prebr{}
{\preindent{2}}\mdToken{Keyword,Extra,Cpp}{function}{\prespace{1}}\mdToken{Identifier,Cpp}{expr}\mdToken{Delimiter,Parenthesis,Cpp,BracketOpen}{(}\mdToken{Identifier,Cpp}{v}\mdToken{Delimiter,Parenthesis,Cpp,BracketClose}{)}{\prespace{1}}\mdToken{Operator,Cpp}{=}{\prespace{1}}\mdToken{Identifier,Cpp}{v}{\prespace{1}}\mdToken{Keyword,Extra,Cpp}{instanceof}{\prespace{1}}\mdToken{Type,Identifier,Cpp}{Symbolic}{\prespace{1}}\mdToken{Operator,Cpp}{?}{\prespace{1}}\mdToken{Identifier,Cpp}{v}\mdToken{Delimiter,Cpp}{.}\mdToken{Identifier,Cpp}{getExpr}\mdToken{Delimiter,Parenthesis,Cpp,BracketOpen}{(}\mdToken{Delimiter,Parenthesis,Cpp,BracketClose}{)}{\prespace{1}}\mdToken{Operator,Cpp}{:}{\prespace{1}}\mdToken{Identifier,Cpp}{v}}%
\end{mdPre}%
\mdHr[class={figureline,madoko},data-line={411}]{}\begin{mdDiv}[data-line={412}]%
% data-line={412}
{}\mdSpan[class={figure-caption}]{\mdSpan[class={caption-before}]{\mdStrong{Figure{\mdNbsp}\mdSpan[class={figure-label}]{5}.} }Type instrumentation to carry both concrete values and symbolic expressions.}% data-line={412}
{}%
\end{mdDiv}%%
\end{mdDiv}%
\begin{mdP}[class={indent,para-continue},data-line={413}]%
% data-line={413}
{}A type such as % data-line={413}
{}\mdSpan[class={math-inline},elem={math-inline}]{$T'$}% data-line={413}
{}  only can be constructed by providing a concrete value % data-line={413}
{}\mdSpan[class={math-inline},elem={math-inline}]{$c$}% data-line={413}
{} of 
type % data-line={414}
{}\mdSpan[class={math-inline},elem={math-inline}]{$T$}% data-line={414}
{} and a symbolic expression % data-line={414}
{}\mdSpan[class={math-inline},elem={math-inline}]{$e$}% data-line={414}
{} to the constructor for % data-line={414}
{}\mdSpan[class={math-inline},elem={math-inline}]{$T'$}% data-line={414}
{}.
This will be done in exactly two places:%
\end{mdP}%
\begin{mdUl}[class={list-star,loose},data-line={417}]%
\begin{mdLi}[data-line={417}]%
\begin{mdP}[data-line={417}]%
% data-line={417}
{}by the creation of symbolic constants associated with the primary
inputs (% data-line={418}
{}\mdSpan[class={math-inline},elem={math-inline}]{$v_1 \ldots v_k$}% data-line={418}
{}) to the program;%
\end{mdP}%%
\end{mdLi}%
\begin{mdLi}[data-line={420}]%
\begin{mdP}[data-line={420}]%
% data-line={420}
{}by the instrumented operations as shown  in Figure% data-line={420}
{}{\mdNbsp}\mdA[class={localref},target-element={figure}]{fig-subtype}{}{\mdSpan[class={figure-label}]{5}}% data-line={420}
{}.%
\end{mdP}%%
\end{mdLi}%%
\end{mdUl}%
\begin{mdP}[data-line={422}]%
% data-line={422}
{}An instrumented operation % data-line={422}
{}\mdSpan[class={math-inline},elem={math-inline}]{$o$}% data-line={422}
{} on arguments (% data-line={422}
{}\mdCode[class={code,code2,language-cpp2,lang-cpp2,cpp2,highlighted},language={cpp2}]{\mdToken{Keyword,Cpp}{this}}% data-line={422}
{}, % data-line={422}
{}\mdCode[class={code,code1,language-python,lang-python,python,highlighted},language={python}]{\mdToken{Identifier,Python}{f1}}% data-line={422}
{}, % data-line={422}
{}{\dots}% data-line={422}
{}, % data-line={422}
{}\mdCode[class={code,code1,language-python,lang-python,python,highlighted},language={python}]{\mdToken{Identifier,Python}{fk}}% data-line={422}
{})
first invokes its corresponding underlying
operator % data-line={424}
{}\mdSpan[class={math-inline},elem={math-inline}]{$T.o$}% data-line={424}
{} on arguments (% data-line={424}
{}\mdCode[class={code,code2,language-cpp2,lang-cpp2,cpp2,highlighted},language={cpp2}]{\mdToken{Keyword,Cpp}{this}}% data-line={424}
{}, % data-line={424}
{}\mdCode[class={code,code1,language-python,lang-python,python,highlighted},language={python}]{\mdToken{Identifier,Python}{f1}}% data-line={424}
{}, % data-line={424}
{}{\dots}% data-line={424}
{}, % data-line={424}
{}\mdCode[class={code,code1,language-python,lang-python,python,highlighted},language={python}]{\mdToken{Identifier,Python}{fk}}% data-line={424}
{}) to get concrete value % data-line={424}
{}\mdCode[class={code,code1,language-python,lang-python,python,highlighted},language={python}]{\mdToken{Identifier,Python}{c}}% data-line={424}
{}.
It then constructs a new expression tree % data-line={425}
{}\mdCode[class={code,code1,language-python,lang-python,python,highlighted},language={python}]{\mdToken{Identifier,Python}{e}}% data-line={425}
{}
rooted at operator % data-line={426}
{}\mdSpan[class={math-inline},elem={math-inline}]{$T.o$}% data-line={426}
{}, whose children are the result of
mapping the function % data-line={427}
{}\mdCode[class={code,code1,language-python,lang-python,python,highlighted},language={python}]{\mdToken{Identifier,Python}{expr}}% data-line={427}
{} over (% data-line={427}
{}\mdCode[class={code,code2,language-cpp2,lang-cpp2,cpp2,highlighted},language={cpp2}]{\mdToken{Keyword,Cpp}{this}}% data-line={427}
{}, % data-line={427}
{}\mdCode[class={code,code1,language-python,lang-python,python,highlighted},language={python}]{\mdToken{Identifier,Python}{f1}}% data-line={427}
{}, % data-line={427}
{}{\dots}% data-line={427}
{}, % data-line={427}
{}\mdCode[class={code,code1,language-python,lang-python,python,highlighted},language={python}]{\mdToken{Identifier,Python}{fk}}% data-line={427}
{}). 
The helper function % data-line={428}
{}\mdCode[class={code,code1,language-python,lang-python,python,highlighted},language={python}]{\mdToken{Identifier,Python}{expr}\mdToken{Delimiter,Parenthesis,Python,BracketOpen}{(}\mdToken{Identifier,Python}{v}\mdToken{Delimiter,Parenthesis,Python,BracketClose}{)}}% data-line={428}
{}
evaluates to an expression tree in the case that % data-line={429}
{}\mdCode[class={code,code1,language-python,lang-python,python,highlighted},language={python}]{\mdToken{Identifier,Python}{v}}% data-line={429}
{} is of % data-line={429}
{}\mdCode[class={code,code2,language-cpp2,lang-cpp2,cpp2,highlighted},language={cpp2}]{\mdToken{Type,Identifier,Cpp}{Symbolic}}% data-line={429}
{} type
(representing a type in % data-line={430}
{}\mdSpan[class={math-inline},elem={math-inline}]{$U'$}% data-line={430}
{}) and evaluates to % data-line={430}
{}\mdCode[class={code,code1,language-python,lang-python,python,highlighted},language={python}]{\mdToken{Identifier,Python}{v}}% data-line={430}
{} itself, a concrete value
of some type in % data-line={431}
{}\mdSpan[class={math-inline},elem={math-inline}]{$U$}% data-line={431}
{}, otherwise.
Finally, having computed the values % data-line={432}
{}\mdCode[class={code,code1,language-python,lang-python,python,highlighted},language={python}]{\mdToken{Identifier,Python}{c}}% data-line={432}
{} and % data-line={432}
{}\mdCode[class={code,code1,language-python,lang-python,python,highlighted},language={python}]{\mdToken{Identifier,Python}{e}}% data-line={432}
{}, the instrumented operator 
returns % data-line={433}
{}\mdCode[class={code,code2,language-cpp2,lang-cpp2,cpp2,highlighted},language={cpp2}]{\mdSpan[class={code-escaped}]{\mdSpan[class={math-inline},elem={math-inline}]{$R'$}}\mdToken{Delimiter,Parenthesis,Cpp,BracketOpen}{(}\mdToken{Identifier,Cpp}{c}\mdToken{Delimiter,Cpp}{,}\mdToken{Identifier,Cpp}{e}\mdToken{Delimiter,Parenthesis,Cpp,BracketClose}{)}}% data-line={433}
{}, where % data-line={433}
{}\mdSpan[class={math-inline},elem={math-inline}]{$R$}% data-line={433}
{} is the return type of operator % data-line={433}
{}\mdSpan[class={math-inline},elem={math-inline}]{$T.o$}% data-line={433}
{},
and % data-line={434}
{}\mdSpan[class={math-inline},elem={math-inline}]{$R'$}% data-line={434}
{} is a subtype of % data-line={434}
{}\mdSpan[class={math-inline},elem={math-inline}]{$R$}% data-line={434}
{} from universe % data-line={434}
{}\mdSpan[class={math-inline},elem={math-inline}]{$U'$}% data-line={434}
{}.%
\end{mdP}%
\begin{mdP}[class={indent},data-line={436}]%
% data-line={436}
{}Looked at another way, the universe % data-line={436}
{}\mdSpan[class={math-inline},elem={math-inline}]{$U'$}% data-line={436}
{} represents the % data-line={436}
{}{\textquotedblleft}tainting{\textquotedblright}% data-line={436}
{} of
types from % data-line={437}
{}\mdSpan[class={math-inline},elem={math-inline}]{$U$}% data-line={437}
{}. Tainted values flow from program inputs to the 
operands of operators. If an operator has been redefined
(as above) then taintedness propagates from its inputs to its outputs.
On the other hand, if the operator has not been redefined, then it will
not propagate taintedness. In the context of DSE, % data-line={441}
{}{\textquotedblleft}taintedness{\textquotedblright}% data-line={441}
{} means
that the instrumented semantics carries along a symbolic expression tree % data-line={442}
{}\mdSpan[class={math-inline},elem={math-inline}]{$e$}% data-line={442}
{}
along with a concrete value % data-line={443}
{}\mdSpan[class={math-inline},elem={math-inline}]{$c$}% data-line={443}
{}.%
\end{mdP}%
\begin{mdP}[class={indent},data-line={445}]%
% data-line={445}
{}The choice of types from the universe % data-line={445}
{}\mdSpan[class={math-inline},elem={math-inline}]{$U'$}% data-line={445}
{} determines how symbolic
expressions are constructed. For each % data-line={446}
{}\mdSpan[class={math-inline},elem={math-inline}]{$T \in U$}% data-line={446}
{}, the % data-line={446}
{}{\textquotedblleft}most symbolic{\textquotedblright}% data-line={446}
{}
(least concrete) choice is the % data-line={447}
{}\mdSpan[class={math-inline},elem={math-inline}]{$T'$}% data-line={447}
{} that redefines every operator of % data-line={447}
{}\mdSpan[class={math-inline},elem={math-inline}]{$T$}% data-line={447}
{}
(as shown in Figure% data-line={448}
{}{\mdNbsp}\mdA[class={localref},target-element={figure}]{fig-subtype}{}{\mdSpan[class={figure-label}]{5}}% data-line={448}
{}).
The % data-line={449}
{}{\textquotedblleft}least symbolic{\textquotedblright}% data-line={449}
{} (most concrete) choice is % data-line={449}
{}\mdSpan[class={math-inline},elem={math-inline}]{$T' = T$}% data-line={449}
{} which
redefines no operators.  Let % data-line={450}
{}\mdSpan[class={math-inline},elem={math-inline}]{$symbolic(T)$}% data-line={450}
{} be the
set of types in % data-line={451}
{}\mdSpan[class={math-inline},elem={math-inline}]{$U'$}% data-line={451}
{} that are subtypes of % data-line={451}
{}\mdSpan[class={math-inline},elem={math-inline}]{$T$}% data-line={451}
{}. The types
in % data-line={452}
{}\mdSpan[class={math-inline},elem={math-inline}]{$symbolic(T)$}% data-line={452}
{} are partially ordered by subset inclusion on the 
set of operators from % data-line={453}
{}\mdSpan[class={math-inline},elem={math-inline}]{$T$}% data-line={453}
{} they redefine.%
\end{mdP}%
\mdHxx[id=sec-sp2dse,label={[3]\{.heading-label\}},toc={},data-line={457},caption={[[3]\{.heading-label\}.{\enspace}]\{.heading-before\}From Strongest Postconditions to DSE},bookmark={3.{\enspace}From Strongest Postconditions to DSE}]{% data-line={457}
{}\mdSpan[class={heading-before}]{\mdSpan[class={heading-label}]{3}.{\enspace}}% data-line={457}
{}From Strongest Postconditions to DSE}\begin{mdP}[data-line={459}]%
% data-line={459}
{}The previous section showed how symbolic expressions can be computed via a
set of instrumented types, where the expressions are computed as a side-effect 
of the execution of program operations.
This section shows how these symbolic expressions can be used to form a 
% data-line={463}
{}\mdEm{path-condition}% data-line={463}
{} (which then can be compiled into a logic formula and 
passed to an automated theorem prover to find new inputs to drive a 
program% data-line={465}
{}{'}% data-line={465}
{}s execution along new paths). We derive a % data-line={465}
{}\mdEm{path-condition}% data-line={465}
{}
directly from the % data-line={466}
{}\mdEm{strongest postcondition}% data-line={466}
{} (symbolic) semantics of our
programming language, refining it to model the basic operations
of an interpreter.%
\end{mdP}%
\mdHxxx[id=sec-strongest-postconditions,label={[3.1]\{.heading-label\}},toc={},data-line={470},caption={[[3.1]\{.heading-label\}.{\enspace}]\{.heading-before\}Strongest Postconditions},bookmark={3.1.{\enspace}Strongest Postconditions}]{% data-line={470}
{}\mdSpan[class={heading-before}]{\mdSpan[class={heading-label}]{3.1}.{\enspace}}% data-line={470}
{}Strongest Postconditions}\begin{mdP}[class={para-continue},data-line={472}]%
% data-line={472}
{}The strongest postcondition transformer % data-line={472}
{}\mdSpan[class={math-inline},elem={math-inline}]{$SP$}% data-line={472}
{}{\mdNbsp}\mdSpan[class={citations},target-element={bibitem}]{[\mdA[class={bibref,localref},target-element={bibitem}]{dijkstra76}{}{\mdSpan[class={bibitem-label}]{6}}]}% data-line={472}
{} is defined over
a predicate % data-line={473}
{}\mdSpan[class={math-inline},elem={math-inline}]{$P$}% data-line={473}
{} representing a set of 
pre-states and a statement % data-line={474}
{}\mdSpan[class={math-inline},elem={math-inline}]{$S$}% data-line={474}
{} from our language. The transformer % data-line={474}
{}\mdSpan[class={math-inline},elem={math-inline}]{$SP(P,S)$}% data-line={474}
{}
yields a predicate % data-line={475}
{}\mdSpan[class={math-inline},elem={math-inline}]{$Q$}% data-line={475}
{} such that for any state % data-line={475}
{}\mdSpan[class={math-inline},elem={math-inline}]{$s$}% data-line={475}
{} satisfying
predicate % data-line={476}
{}\mdSpan[class={math-inline},elem={math-inline}]{$P$}% data-line={476}
{}, the execution of statement % data-line={476}
{}\mdSpan[class={math-inline},elem={math-inline}]{$S$}% data-line={476}
{} from state % data-line={476}
{}\mdSpan[class={math-inline},elem={math-inline}]{$s$}% data-line={476}
{},
if it does not go wrong or diverge, yields a state % data-line={477}
{}\mdSpan[class={math-inline},elem={math-inline}]{$s'$}% data-line={477}
{} satisfying predicate % data-line={477}
{}\mdSpan[class={math-inline},elem={math-inline}]{$Q$}% data-line={477}
{}. 
The strongest postcondition for the statements in our language
is defined by the following five rules:%
\end{mdP}%
\begin{mdDiv}[class={mathpre,para-block,input-mathpre},elem={mathpre},data-line={481}]%
\begin{mdDiv}[class={math-display}]%
\[% data-line={482}
\begin{mdMathprearray}%
1.\mdMathspace{2}&\mdMathspace{1}\mathid{SP}(\mathid{P},\mdMathspace{1}\mathid{x}\mdMathspace{1}:=\mdMathspace{1}\mathid{E})\mdMathspace{1}{~\buildrel\triangle\over=~}\mdMathspace{1}\exists \mathid{y}\mdMathspace{1}.\mdMathspace{1}(\mathid{x}\mdMathspace{1}=\mdMathspace{1}\mathid{E}\mdMathspace{1}[\mdMathspace{1}\mathid{x}\mdMathspace{1}\rightarrow \mathid{y}\mdMathspace{1}])\mdMathspace{1}\wedge \mathid{P}\mdMathspace{1}[\mdMathspace{1}\mathid{x}\mdMathspace{1}\rightarrow \mathid{y}\mdMathspace{1}]\mdMathbr{}
2.\mdMathspace{2}&\mdMathspace{1}\mathid{SP}(\mathid{P},\mdMathspace{1}\mathkw{skip})\mdMathspace{1}{~\buildrel\triangle\over=~}\mdMathspace{1}\mathid{P}\mdMathbr{}
3.\mdMathspace{2}&\mdMathspace{1}\mathid{SP}(\mathid{P},\mathid{S}_{1};\mathid{S}_{2})\mdMathspace{1}{~\buildrel\triangle\over=~}\mdMathspace{1}\mathid{SP}(\mathid{SP}(\mathid{P},\mathid{S}_{1}),\mdMathspace{1}\mathid{S}_{2})\mdMathbr{}
4.\mdMathspace{2}&\mdMathspace{1}\mathid{SP}(\mathid{P},\mathkw{if}\mdMathspace{1}\mathid{E}\mdMathspace{1}\mathkw{then}\mdMathspace{1}\mathid{S}_1\mdMathspace{1}\mathkw{else}\mdMathspace{1}\mathid{S}_2\mdMathspace{1}\mathkw{end})\mdMathspace{1}{~\buildrel\triangle\over=~}\mdMathbr{}
\mdMathindent{4}&\mdMathspace{7}\mathid{SP}(\mathid{P}\wedge \mathid{E},\mathid{S}_1)\mdMathspace{1}\vee  \mathid{SP}(\mathid{P}\mdMathspace{1}\wedge \neg \mathid{E},\mathid{S}_2)\mdMathbr{}
5.\mdMathspace{2}&\mdMathspace{1}\mathid{SP}(\mathid{P},\mathkw{while}\mdMathspace{1}\mathid{E}\mdMathspace{1}\mathkw{do}\mdMathspace{1}\mathid{S}\mdMathspace{1}\mathkw{end})\mdMathspace{1}{~\buildrel\triangle\over=~}\mdMathbr{}
\mdMathindent{4}&\mdMathspace{6}\mathid{SP}(\mathid{P},\mathkw{if}\mdMathspace{1}\mathid{E}\mdMathspace{1}\mathkw{then}\mdMathspace{1}\mathid{S};\mdMathspace{1}\mathkw{while}\mdMathspace{1}\mathid{E}\mdMathspace{1}\mathkw{do}\mdMathspace{1}\mathid{S}\mdMathspace{1}\mathkw{end}\mdMathspace{1}\mathkw{else}\mdMathspace{1}\mathkw{skip}\mdMathspace{1}\mathkw{end})
\end{mdMathprearray}\]%
\end{mdDiv}%%
\end{mdDiv}%
\begin{mdP}[data-line={491}]%
% data-line={491}
{}Rule (1) defines the strongest postcondition for 
the assignment statement. The assignment is modelled
logically by the equality % data-line={493}
{}\mdSpan[class={math-inline},elem={math-inline}]{$x = E$}% data-line={493}
{} where any free occurrence of % data-line={493}
{}\mdSpan[class={math-inline},elem={math-inline}]{$x$}% data-line={493}
{} in % data-line={493}
{}\mdSpan[class={math-inline},elem={math-inline}]{$E$}% data-line={493}
{}
is replaced by the existentially quantified variable % data-line={494}
{}\mdSpan[class={math-inline},elem={math-inline}]{$y$}% data-line={494}
{}, which represents
the value of % data-line={495}
{}\mdSpan[class={math-inline},elem={math-inline}]{$x$}% data-line={495}
{} in the pre-state. The same substitution (% data-line={495}
{}\mdSpan[class={math-inline},elem={math-inline}]{$[x \rightarrow y ]$}% data-line={495}
{})
is applied to the pre-state predicate % data-line={496}
{}\mdSpan[class={math-inline},elem={math-inline}]{$P$}% data-line={496}
{}.%
\end{mdP}%
\begin{mdP}[class={indent},data-line={498}]%
% data-line={498}
{}Rules (2)-(5) define the strongest postcondition for the four control-flow
statements. The rules for the % data-line={499}
{}\mdStrong{skip}% data-line={499}
{} statement and sequencing (;) are
straightforward. 
Of particular interest, note that the rule for the % data-line={501}
{}\mdStrong{if-then-else}% data-line={501}
{} 
statement splits cases on the expression % data-line={502}
{}\mdSpan[class={math-inline},elem={math-inline}]{$E$}% data-line={502}
{}. 
It is here that DSE will choose one of the cases for us, as the concrete
execution will evaluate % data-line={504}
{}\mdSpan[class={math-inline},elem={math-inline}]{$E$}% data-line={504}
{} either to be true or false. This gives rise
to the path-condition (either % data-line={505}
{}\mdSpan[class={math-inline},elem={math-inline}]{$P \wedge E$}% data-line={505}
{} or % data-line={505}
{}\mdSpan[class={math-inline},elem={math-inline}]{$P \wedge \neg E$}% data-line={505}
{}). The
recursive rule for the % data-line={506}
{}\mdStrong{while}% data-line={506}
{} loop unfolds as many times as the
expression % data-line={507}
{}\mdSpan[class={math-inline},elem={math-inline}]{$E$}% data-line={507}
{} evaluates true, adding to the path-condition.%
\end{mdP}%
\mdHxxx[id=sp-refined,label={[3.2]\{.heading-label\}},toc={},data-line={509},caption={[[3.2]\{.heading-label\}.{\enspace}]\{.heading-before\}From \$SP\$ to DSE},bookmark={3.2.{\enspace}From \$SP\$ to DSE}]{% data-line={509}
{}\mdSpan[class={heading-before}]{\mdSpan[class={heading-label}]{3.2}.{\enspace}}% data-line={509}
{}From % data-line={509}
{}\mdSpan[class={math-inline},elem={math-inline}]{$SP$}% data-line={509}
{} to DSE}\begin{mdP}[data-line={511}]%
% data-line={511}
{}Assume that an execution begins with the assignment of initial values % data-line={511}
{}\mdSpan[class={math-inline},elem={math-inline}]{$c_1 \ldots c_k$}% data-line={511}
{} to the program % data-line={511}
{}\mdSpan[class={math-inline},elem={math-inline}]{$P$}% data-line={511}
{}{'}% data-line={511}
{}s inputs
% data-line={512}
{}\mdSpan[class={math-inline},elem={math-inline}]{$V = \{ v_1 : T_1 \ldots v_k : T_k \}$}% data-line={512}
{}.  To seed symbolic execution, some of the types % data-line={512}
{}\mdSpan[class={math-inline},elem={math-inline}]{$T_i$}% data-line={512}
{} are
replaced by symbolic counterparts % data-line={513}
{}\mdSpan[class={math-inline},elem={math-inline}]{$T'_i$}% data-line={513}
{}, in which case
% data-line={514}
{}\mdSpan[class={math-inline},elem={math-inline}]{$v_i$}% data-line={514}
{} is initialized to the value % data-line={514}
{}\mdSpan[class={math-inline},elem={math-inline}]{$sc_i = T'_i (c_i,SC(v_i))$}% data-line={514}
{} instead of the value % data-line={514}
{}\mdSpan[class={math-inline},elem={math-inline}]{$c_i$}% data-line={514}
{},
where % data-line={515}
{}\mdSpan[class={math-inline},elem={math-inline}]{$SC(v_i)$}% data-line={515}
{} is the symbolic constant representing the initial value of variable % data-line={515}
{}\mdSpan[class={math-inline},elem={math-inline}]{$v_i$}% data-line={515}
{}.
The symbolic constant % data-line={516}
{}\mdSpan[class={math-inline},elem={math-inline}]{$SC(v_i)$}% data-line={516}
{} can be thought of as representing any value of
type % data-line={517}
{}\mdSpan[class={math-inline},elem={math-inline}]{$T_i$}% data-line={517}
{}, which includes the value % data-line={517}
{}\mdSpan[class={math-inline},elem={math-inline}]{$c_i$}% data-line={517}
{}.% data-line={517}
{} % data-line={517}
{}%
\end{mdP}%
\begin{mdP}[class={indent,para-continue},data-line={519}]%
% data-line={519}
{}Let % data-line={519}
{}\mdSpan[class={math-inline},elem={math-inline}]{$V_s$}% data-line={519}
{} and % data-line={519}
{}\mdSpan[class={math-inline},elem={math-inline}]{$V_c$}% data-line={519}
{} partition the variables of % data-line={519}
{}\mdSpan[class={math-inline},elem={math-inline}]{$V$}% data-line={519}
{} into those variables
that are treated symbolically (% data-line={520}
{}\mdSpan[class={math-inline},elem={math-inline}]{$V_s$}% data-line={520}
{}) and those that are treated concretely 
(% data-line={521}
{}\mdSpan[class={math-inline},elem={math-inline}]{$V_c$}% data-line={521}
{}). The initial state of the program is characterized by the formula%
\end{mdP}%
\begin{mdDiv}[class={equation,align-center,para-block},label={[(1)]\{.equation-label\}},elem={equation},line-adjust={0},data-line={523}]%
% data-line={523}
{}\mdSpan[class={equation-before}]{\mdSpan[class={equation-label}]{(1)}}% data-line={523}
{}
\begin{mdDiv}[class={math,para-block,input-math},elem={math},color={},math-needpdf={},line-adjust={0},data-line={524}]%
\begin{mdDiv}[class={math-display}]%
\[% data-line={524}
Init = (\bigwedge_{v_i \in V_s} v_i = sc_i) ~\wedge~ (~\bigwedge_{v_i \in V_c} v_i = c_i)\]%
\end{mdDiv}%%
\end{mdDiv}%%
\end{mdDiv}%
\begin{mdP}[data-line={527}]%
% data-line={527}
{}Thus, we see that the initial value of every input variable is characterized by 
a symbolic constant % data-line={528}
{}\mdSpan[class={math-inline},elem={math-inline}]{$sc_i$}% data-line={528}
{} or constant % data-line={528}
{}\mdSpan[class={math-inline},elem={math-inline}]{$c_i$}% data-line={528}
{}.  We assume that every non-input
variable in the program is initialized before being used.%
\end{mdP}%
\begin{mdP}[class={indent},data-line={531}]%
% data-line={531}
{}The strongest postcondition is formulated to deal with open programs,
programs in which some variables are used before being assigned to. 
This surfaces in Rule (1) for assignment, which uses existential quantification
to refer to the value of variable % data-line={534}
{}\mdSpan[class={math-inline},elem={math-inline}]{$x$}% data-line={534}
{} in the pre-state.%
\end{mdP}%
\begin{mdP}[class={indent,para-continue},data-line={536}]%
% data-line={536}
{}By construction,
we have that every variable is defined before being used.  This means
that the precondition % data-line={538}
{}\mdSpan[class={math-inline},elem={math-inline}]{$P$}% data-line={538}
{} can be reformulated as a pair % data-line={538}
{}\mdSpan[class={math-inline},elem={math-inline}]{$<\sigma,P_c>$}% data-line={538}
{},
where % data-line={539}
{}\mdSpan[class={math-inline},elem={math-inline}]{$\sigma$}% data-line={539}
{} is a store mapping variables to values and % data-line={539}
{}\mdSpan[class={math-inline},elem={math-inline}]{$P_c$}% data-line={539}
{} is
the path-condition, a list of symbolic expressions (predicates) 
corresponding
to the expressions % data-line={542}
{}\mdSpan[class={math-inline},elem={math-inline}]{$E$}% data-line={542}
{} evaluated in the context of an % data-line={542}
{}\mdStrong{if-then-else}% data-line={542}
{} 
statement. Initially, we have that :%
\end{mdP}%
\begin{mdDiv}[class={equation,align-center,para-block},label={[(2)]\{.equation-label\}},elem={equation},line-adjust={0},data-line={545}]%
% data-line={545}
{}\mdSpan[class={equation-before}]{\mdSpan[class={equation-label}]{(2)}}% data-line={545}
{}
\begin{mdDiv}[class={math,para-block,input-math},elem={math},color={},math-needpdf={},line-adjust={0},data-line={546}]%
\begin{mdDiv}[class={math-display}]%
\[% data-line={546}
\sigma = \{ (v_i,sc_i) | v_i \in V_s \} \cup \{ (v_i,c_i) | v_i \in V_c \}\]%
\end{mdDiv}%%
\end{mdDiv}%%
\end{mdDiv}%
\begin{mdP}[class={para-continue},data-line={549}]%
% data-line={549}
{}representing the initial condition % data-line={549}
{}\mdSpan[class={math-inline},elem={math-inline}]{$Init$}% data-line={549}
{}, and % data-line={549}
{}\mdSpan[class={math-inline},elem={math-inline}]{$P_c = []$}% data-line={549}
{}, the empty list.
We use % data-line={550}
{}\mdSpan[class={math-inline},elem={math-inline}]{$\sigma'$}% data-line={550}
{} to refer to the formula that the store % data-line={550}
{}\mdSpan[class={math-inline},elem={math-inline}]{$\sigma$}% data-line={550}
{}
induces:%
\end{mdP}%
\begin{mdDiv}[class={equation,align-center,para-block},label={[(3)]\{.equation-label\}},elem={equation},line-adjust={0},data-line={553}]%
% data-line={553}
{}\mdSpan[class={equation-before}]{\mdSpan[class={equation-label}]{(3)}}% data-line={553}
{}
\begin{mdDiv}[class={math,para-block,input-math},elem={math},color={},math-needpdf={},line-adjust={0},data-line={554}]%
\begin{mdDiv}[class={math-display}]%
\[% data-line={554}
\sigma ' =  \bigwedge_{(v,V) \in \sigma} (v = V)\]%
\end{mdDiv}%%
\end{mdDiv}%%
\end{mdDiv}%
\begin{mdP}[data-line={557}]%
% data-line={557}
{}Thus, the pair % data-line={557}
{}\mdSpan[class={math-inline},elem={math-inline}]{$<\sigma,P_c>$}% data-line={557}
{} represents the predicate 
% data-line={558}
{}\mdSpan[class={math-inline},elem={math-inline}]{$P = \sigma' \wedge (\bigwedge_{c \in P_c} c)$}% data-line={558}
{}.
A store % data-line={559}
{}\mdSpan[class={math-inline},elem={math-inline}]{$\sigma$}% data-line={559}
{} supports two operations: % data-line={559}
{}\mdSpan[class={math-inline},elem={math-inline}]{$\sigma[x]$}% data-line={559}
{} which denotes the
value that % data-line={560}
{}\mdSpan[class={math-inline},elem={math-inline}]{$x$}% data-line={560}
{} maps to under % data-line={560}
{}\mdSpan[class={math-inline},elem={math-inline}]{$\sigma$}% data-line={560}
{}; % data-line={560}
{}\mdSpan[class={math-inline},elem={math-inline}]{$\sigma[x \mapsto V]$}% data-line={560}
{}, which 
produces a new store in which % data-line={561}
{}\mdSpan[class={math-inline},elem={math-inline}]{$x$}% data-line={561}
{} maps to value % data-line={561}
{}\mdSpan[class={math-inline},elem={math-inline}]{$V$}% data-line={561}
{} and is everywhere 
else the same as % data-line={562}
{}\mdSpan[class={math-inline},elem={math-inline}]{$\sigma$}% data-line={562}
{}.%
\end{mdP}%
\begin{mdP}[class={indent,para-continue},data-line={564}]%
% data-line={564}
{}Now, we can redefine strongest postcondition for assignment to eliminate the
use of existential quantification and model the operation of an interpreter,
by separating out the notion of the store:%
\end{mdP}%
\begin{mdDiv}[class={mathpre,para-block,input-mathpre},elem={mathpre},data-line={568}]%
\begin{mdDiv}[class={math-display}]%
\[% data-line={569}
\begin{mdMathprearray}%
1.\mdMathspace{1}&\mdMathspace{1}\mathid{SP}(<\sigma,\mdMathspace{1}\mathid{P}_\mathid{c}>,\mdMathspace{1}\mathid{x}\mdMathspace{1}:=\mdMathspace{1}\mathid{E})\mdMathspace{1}{~\buildrel\triangle\over=~}\mdMathspace{1}<\sigma[\mathid{x}\mdMathspace{1}\mapsto \mathid{eval}(\sigma,\mathid{E})],\mdMathspace{1}\mathid{P}_\mathid{c}>\\
\end{mdMathprearray}\]%
\end{mdDiv}%%
\end{mdDiv}%
\begin{mdP}[data-line={572}]%
% data-line={572}
{}where % data-line={572}
{}\mdSpan[class={math-inline},elem={math-inline}]{$eval(\sigma,E)$}% data-line={572}
{} evaluates expression % data-line={572}
{}\mdSpan[class={math-inline},elem={math-inline}]{$E$}% data-line={572}
{} under the store % data-line={572}
{}\mdSpan[class={math-inline},elem={math-inline}]{$\sigma$}% data-line={572}
{}
(where every occurrence of a free variable
% data-line={574}
{}\mdSpan[class={math-inline},elem={math-inline}]{$v$}% data-line={574}
{} in % data-line={574}
{}\mdSpan[class={math-inline},elem={math-inline}]{$E$}% data-line={574}
{} is replaced by the value % data-line={574}
{}\mdSpan[class={math-inline},elem={math-inline}]{$\sigma[v]$}% data-line={574}
{}). 
This is the standard substitution rule of a standard operational semantics.%
\end{mdP}%
\begin{mdP}[class={indent,para-continue},data-line={577}]%
% data-line={577}
{}We also redefine the rule for the % data-line={577}
{}\mdStrong{if-then-else}% data-line={577}
{} statement so
that it chooses which branch to take and appends the appropriate
symbolic expression (predicate) to the path-condition % data-line={579}
{}\mdSpan[class={math-inline},elem={math-inline}]{$P_c$}% data-line={579}
{}:%
\end{mdP}%
\begin{mdDiv}[class={mathpre,para-block,input-mathpre},elem={mathpre},data-line={581}]%
\begin{mdDiv}[class={math-display}]%
\[% data-line={582}
\begin{mdMathprearray}%
4.\mdMathspace{1}&\mdMathspace{1}\mathid{SP}(<\sigma,\mdMathspace{1}\mathid{P}_\mathid{c}>,\mdMathspace{1}\mathkw{if}\mdMathspace{1}\mathid{E}\mdMathspace{1}\mathkw{then}\mdMathspace{1}\mathid{S}_1\mdMathspace{1}\mathkw{else}\mdMathspace{1}\mathid{S}_2\mdMathspace{1}\mathkw{end})\mdMathspace{1}{~\buildrel\triangle\over=~}\mdMathspace{2}\mdMathbr{}
\mdMathindent{3}&\mdMathspace{3}\mathkw{let}\mdMathspace{1}\mathid{choice}\mdMathspace{1}=\mdMathspace{1}\mathid{eval}(\sigma,\mathid{E})\mdMathspace{1}\mathkw{in}\mdMathbr{}
\mdMathindent{3}&\mdMathspace{3}\mathkw{if}\mdMathspace{1}\mathid{choice}\mdMathspace{1}\mathkw{then}\mdMathspace{1}\mathid{SP}(<\sigma,\mdMathspace{1}\mathid{P}_\mathid{c}\mdMathspace{1}::\mdMathspace{1}\mathid{expr}(\mathid{choice})\mdMathspace{1}>,\mathid{S}_1)\mdMathspace{1}\mdMathbr{}
\mdMathindent{3}&\mdMathspace{3}\mathkw{else}\mdMathspace{1}\mathid{SP}(<\sigma,\mdMathspace{1}\mathid{P}_\mathid{c}\mdMathspace{1}::\mdMathspace{1}\neg \mathid{expr}(\mathid{choice})\mdMathspace{1}>,\mathid{S}_2)
\end{mdMathprearray}\]%
\end{mdDiv}%%
\end{mdDiv}%
\begin{mdP}[data-line={588}]%
% data-line={588}
{}The other strongest postcondition rules remain unchanged.%
\end{mdP}%
\mdHxxx[id=sec-summing-it-up,label={[3.3]\{.heading-label\}},toc={},data-line={590},caption={[[3.3]\{.heading-label\}.{\enspace}]\{.heading-before\}Summing it up},bookmark={3.3.{\enspace}Summing it up}]{% data-line={590}
{}\mdSpan[class={heading-before}]{\mdSpan[class={heading-label}]{3.3}.{\enspace}}% data-line={590}
{}Summing it up}\begin{mdP}[data-line={592}]%
% data-line={592}
{}We have shown how the symbolic predicate transformer % data-line={592}
{}\mdSpan[class={math-inline},elem={math-inline}]{$SP$}% data-line={592}
{} can 
be refined into a symbolic interpreter operating over the
symbolic types defined in the previous section.
In the case when every input variable is symbolic and every
operator is redefined, the path-condition is equivalent to the
% data-line={597}
{}\mdEm{strongest postcondition}% data-line={597}
{} of the execution path % data-line={597}
{}\mdSpan[class={math-inline},elem={math-inline}]{$p$}% data-line={597}
{}. 
This guarantees that the path-condition for % data-line={598}
{}\mdSpan[class={math-inline},elem={math-inline}]{$p$}% data-line={598}
{} is % data-line={598}
{}\mdEm{sound}% data-line={598}
{}.
In the case where a subset of the input variables are symbolic
and/or not all operators are redefined, the path-condition of % data-line={600}
{}\mdSpan[class={math-inline},elem={math-inline}]{$p$}% data-line={600}
{} 
is not guaranteed to be sound. We leave it as an exercise to the
reader to establish sufficient conditions under which the 
use of concrete values in place of symbolic expressions is 
guaranteed to result in sound path-conditions.%
\end{mdP}%
\begin{mdP}[class={indent},data-line={606}]%
% data-line={606}
{}This section does not address the compilation of a symbolic
expression to the (logic) language of an underlying ATP, nor the
lifting of a satisfying assignment to a formula back to the
level of the source language. This is best done for a particular
source language and ATP, as detailed in the next section.%
\end{mdP}%
\mdHxx[id=sec-impl,label={[4]\{.heading-label\}},toc={},data-line={613},caption={[[4]\{.heading-label\}.{\enspace}]\{.heading-before\}Architecture of PyExZ3},bookmark={4.{\enspace}Architecture of PyExZ3}]{% data-line={613}
{}\mdSpan[class={heading-before}]{\mdSpan[class={heading-label}]{4}.{\enspace}}% data-line={613}
{}Architecture of PyExZ3}\begin{mdP}[data-line={615}]%
% data-line={615}
{}In this section we present the high-level architecture
of a simple DSE tool for the Python language, written in Python, called% data-line={616}
{}{\mdNbsp}\mdA{https://github.com/thomasjball/PyExZ3/}{}{PyExZ3}% data-line={616}
{}. 
Figure% data-line={617}
{}{\mdNbsp}\mdA[class={localref},target-element={figure}]{fig-arch}{}{\mdSpan[class={figure-label}]{6}}% data-line={617}
{}
shows the class diagram (dashed edges are % data-line={618}
{}{\textquotedblleft}has-a{\textquotedblright}% data-line={618}
{} relationships; solid edges
are % data-line={619}
{}{\textquotedblleft}is-a{\textquotedblright}% data-line={619}
{} relationships) of the tool.%
\end{mdP}%
\begin{mdDiv}[class={figure,align-center},id=fig-arch,label={[6]\{.figure-label\}},elem={figure},toc={tof},toc-line={[6]\{.figure-label\}. Classes in PyExZ3},caption={Classes in PyExZ3},page-align={here},data-line={621}]%
\begin{mdP}[data-line={622}]%
% data-line={622}
{}\mdImg[width={1.00\linewidth}]{arch.png}% data-line={622}
{}%
\end{mdP}%
\mdHr[class={figureline,madoko},data-line={623}]{}\begin{mdDiv}[data-line={624}]%
% data-line={624}
{}\mdSpan[class={figure-caption}]{\mdSpan[class={caption-before}]{\mdStrong{Figure{\mdNbsp}\mdSpan[class={figure-label}]{6}.} }Classes in PyExZ3}% data-line={624}
{}%
\end{mdDiv}%%
\end{mdDiv}%
\mdHxxx[id=sec-loading-the-code-under-test,label={[4.1]\{.heading-label\}},toc={},data-line={627},caption={[[4.1]\{.heading-label\}.{\enspace}]\{.heading-before\}Loading the code under test},bookmark={4.1.{\enspace}Loading the code under test}]{% data-line={627}
{}\mdSpan[class={heading-before}]{\mdSpan[class={heading-label}]{4.1}.{\enspace}}% data-line={627}
{}Loading the code under test}\begin{mdP}[data-line={629}]%
% data-line={629}
{}The % data-line={629}
{}\mdCode[class={code,code1,language-python,lang-python,python,highlighted},language={python}]{\mdToken{Namespace,Identifier,Python}{Loader}}% data-line={629}
{} class takes as input the name of a Python file (e.g., % data-line={629}
{}\mdCode[class={code,code1,language-python,lang-python,python,highlighted},language={python}]{\mdToken{Identifier,Python}{foo}\mdToken{Delimiter,Python}{.}\mdToken{Identifier,Python}{py}}% data-line={629}
{}) 
to import. The loader expects to find a function named
% data-line={631}
{}\mdCode[class={code,code1,language-python,lang-python,python,highlighted},language={python}]{\mdToken{Identifier,Python}{foo}}% data-line={631}
{} inside the file % data-line={631}
{}\mdCode[class={code,code1,language-python,lang-python,python,highlighted},language={python}]{\mdToken{Identifier,Python}{foo}\mdToken{Delimiter,Python}{.}\mdToken{Identifier,Python}{py}}% data-line={631}
{}, which will serve as the starting point
for symbolic execution. The % data-line={632}
{}\mdCode[class={code,code1,language-python,lang-python,python,highlighted},language={python}]{\mdToken{Namespace,Identifier,Python}{FunctionInvocation}}% data-line={632}
{} class
wraps this starting point. By default, each parameter to % data-line={633}
{}\mdCode[class={code,code1,language-python,lang-python,python,highlighted},language={python}]{\mdToken{Identifier,Python}{foo}}% data-line={633}
{} is
a % data-line={634}
{}\mdCode[class={code,code1,language-python,lang-python,python,highlighted},language={python}]{\mdToken{Namespace,Identifier,Python}{SymbolicInteger}}% data-line={634}
{} unless there is decorator % data-line={634}
{}\mdCode[class={code,code1,language-python,lang-python,python,highlighted},language={python}]{\mdToken{Delimiter,Python}{@}\mdToken{Identifier,Python}{symbolic}}% data-line={634}
{} specifying
the type to use for a particular argument.%
\end{mdP}%
\begin{mdP}[class={indent},data-line={637}]%
% data-line={637}
{}The loader provides the capability to reload the
module % data-line={638}
{}\mdCode[class={code,code1,language-python,lang-python,python,highlighted},language={python}]{\mdToken{Identifier,Python}{foo}\mdToken{Delimiter,Python}{.}\mdToken{Identifier,Python}{py}}% data-line={638}
{} so that the function % data-line={638}
{}\mdCode[class={code,code1,language-python,lang-python,python,highlighted},language={python}]{\mdToken{Identifier,Python}{foo}}% data-line={638}
{} can be 
reexecuted within the same process from the same initial
state with different inputs (see the class % data-line={640}
{}\mdCode[class={code,code1,language-python,lang-python,python,highlighted},language={python}]{\mdToken{Namespace,Identifier,Python}{ExplorationEngine}}% data-line={640}
{})
via the % data-line={641}
{}\mdCode[class={code,code1,language-python,lang-python,python,highlighted},language={python}]{\mdToken{Namespace,Identifier,Python}{FunctionInvocation}}% data-line={641}
{} class.%
\end{mdP}%
\begin{mdP}[class={indent},data-line={643}]%
% data-line={643}
{}Finally, the loader looks for specially named functions % data-line={643}
{}\mdCode[class={code,code1,language-python,lang-python,python,highlighted},language={python}]{\mdToken{Identifier,Python}{expected\_result}}% data-line={643}
{}
(% data-line={644}
{}\mdCode[class={code,code1,language-python,lang-python,python,highlighted},language={python}]{\mdToken{Identifier,Python}{expected\_result\_set}}% data-line={644}
{}) in file % data-line={644}
{}\mdCode[class={code,code1,language-python,lang-python,python,highlighted},language={python}]{\mdToken{Identifier,Python}{foo}\mdToken{Delimiter,Python}{.}\mdToken{Identifier,Python}{py}}% data-line={644}
{} to use as a test oracle after
the path exploration (by % data-line={645}
{}\mdCode[class={code,code1,language-python,lang-python,python,highlighted},language={python}]{\mdToken{Namespace,Identifier,Python}{ExplorationEngine}}% data-line={645}
{}) has completed. These
functions are expected to return a list of values to check
against the list of return values collected from the executions of 
the % data-line={648}
{}\mdCode[class={code,code1,language-python,lang-python,python,highlighted},language={python}]{\mdToken{Identifier,Python}{foo}}% data-line={648}
{} function.
The presence of the function % data-line={649}
{}\mdCode[class={code,code1,language-python,lang-python,python,highlighted},language={python}]{\mdToken{Identifier,Python}{expected\_result}}% data-line={649}
{} (% data-line={649}
{}\mdCode[class={code,code1,language-python,lang-python,python,highlighted},language={python}]{\mdToken{Identifier,Python}{expected\_result\_set}}% data-line={649}
{}) 
yields a comparison of the two lists as bags (sets). We use such weaker
tests, rather than list equality, because the order in which paths
are explored by the % data-line={652}
{}\mdCode[class={code,code1,language-python,lang-python,python,highlighted},language={python}]{\mdToken{Namespace,Identifier,Python}{ExplorationEngine}}% data-line={652}
{} can easily change due to small
differences in the input programs.%
\end{mdP}%
\mdHxxx[id=sec-symbolic-types,label={[4.2]\{.heading-label\}},toc={},data-line={655},caption={[[4.2]\{.heading-label\}.{\enspace}]\{.heading-before\}Symbolic types},bookmark={4.2.{\enspace}Symbolic types}]{% data-line={655}
{}\mdSpan[class={heading-before}]{\mdSpan[class={heading-label}]{4.2}.{\enspace}}% data-line={655}
{}Symbolic types}\begin{mdP}[data-line={657}]%
% data-line={657}
{}Python supports multiple inheritance and, more importantly,
allows user-defined classes to inherit 
from its built-in types (such as % data-line={659}
{}\mdCode[class={code,code1,language-python,lang-python,python,highlighted},language={python}]{\mdToken{Identifier,Python}{object}}% data-line={659}
{} and % data-line={659}
{}\mdCode[class={code,code1,language-python,lang-python,python,highlighted},language={python}]{\mdToken{Identifier,Python}{int}}% data-line={659}
{}).
We use these two features two implement
symbolic versions of Python objects and integers, 
following the instrumented type approach defined in Section% data-line={662}
{}{\mdNbsp}\mdA[class={localref},target-element={h1}]{sec-semantics}{}{\mdSpan[class={heading-label}]{2}}% data-line={662}
{}.%
\end{mdP}%
\begin{mdP}[class={indent},data-line={664}]%
% data-line={664}
{}The abstract class % data-line={664}
{}\mdCode[class={code,code1,language-python,lang-python,python,highlighted},language={python}]{\mdToken{Namespace,Identifier,Python}{SymbolicType}}% data-line={664}
{} contains the 
symbolic expression tree and provides basic functions for constructing 
and accessing the tree.  This class does double duty, as it is used
to represent the (typed) symbolic constants associated with the parameters to the 
function, as well as the expression trees (per Section% data-line={668}
{}{\mdNbsp}\mdA[class={localref},target-element={h1}]{sec-semantics}{}{\mdSpan[class={heading-label}]{2}}% data-line={668}
{}). Recall
that the symbolic constants only appear as leaves of expression trees.
This means that the expression tree stored in a % data-line={670}
{}\mdCode[class={code,code1,language-python,lang-python,python,highlighted},language={python}]{\mdToken{Namespace,Identifier,Python}{SymbolicType}}% data-line={670}
{} will
have instances of a % data-line={671}
{}\mdCode[class={code,code1,language-python,lang-python,python,highlighted},language={python}]{\mdToken{Namespace,Identifier,Python}{SymbolicType}}% data-line={671}
{} as some of its leaves, namely
those leaves representing the symbolic constants.
% data-line={673}
{}% data-line={673}
{}
The abstract class provides an % data-line={674}
{}\mdCode[class={code,code1,language-python,lang-python,python,highlighted},language={python}]{\mdToken{Identifier,Python}{unwrap}}% data-line={674}
{} method which returns
the pair of concrete value and expression tree associated with
the % data-line={676}
{}\mdCode[class={code,code1,language-python,lang-python,python,highlighted},language={python}]{\mdToken{Namespace,Identifier,Python}{SymbolicType}}% data-line={676}
{}, as well as a % data-line={676}
{}\mdCode[class={code,code1,language-python,lang-python,python,highlighted},language={python}]{\mdToken{Identifier,Python}{wrap}}% data-line={676}
{} method that takes a 
pair of concrete value and expression tree and creates a % data-line={677}
{}\mdCode[class={code,code1,language-python,lang-python,python,highlighted},language={python}]{\mdToken{Namespace,Identifier,Python}{SymbolicType}}% data-line={677}
{}
encapsulating them.%
\end{mdP}%
\begin{mdP}[class={indent},data-line={680}]%
% data-line={680}
{}The class % data-line={680}
{}\mdCode[class={code,code1,language-python,lang-python,python,highlighted},language={python}]{\mdToken{Namespace,Identifier,Python}{SymbolicObject}}% data-line={680}
{} inherits from both % data-line={680}
{}\mdCode[class={code,code1,language-python,lang-python,python,highlighted},language={python}]{\mdToken{Identifier,Python}{object}}% data-line={680}
{} and % data-line={680}
{}\mdCode[class={code,code1,language-python,lang-python,python,highlighted},language={python}]{\mdToken{Namespace,Identifier,Python}{SymbolicType}}% data-line={680}
{} and
overrides the basic comparison operations (% data-line={681}
{}\mdCode[class={code,code1,language-python,lang-python,python,highlighted},language={python}]{\mdToken{Predefined,Python}{\_\_eq\_\_}}% data-line={681}
{}, % data-line={681}
{}\mdCode[class={code,code1,language-python,lang-python,python,highlighted},language={python}]{\mdToken{Predefined,Python}{\_\_neq\_\_}}% data-line={681}
{}, % data-line={681}
{}\mdCode[class={code,code1,language-python,lang-python,python,highlighted},language={python}]{\mdToken{Predefined,Python}{\_\_lt\_\_}}% data-line={681}
{}, % data-line={681}
{}\mdCode[class={code,code1,language-python,lang-python,python,highlighted},language={python}]{\mdToken{Predefined,Python}{\_\_le\_\_}}% data-line={681}
{},
% data-line={682}
{}\mdCode[class={code,code1,language-python,lang-python,python,highlighted},language={python}]{\mdToken{Predefined,Python}{\_\_gt\_\_}}% data-line={682}
{}, and % data-line={682}
{}\mdCode[class={code,code1,language-python,lang-python,python,highlighted},language={python}]{\mdToken{Predefined,Python}{\_\_ge\_\_}}% data-line={682}
{}).
The class % data-line={683}
{}\mdCode[class={code,code1,language-python,lang-python,python,highlighted},language={python}]{\mdToken{Namespace,Identifier,Python}{SymbolicInteger}}% data-line={683}
{} inherits from both % data-line={683}
{}\mdCode[class={code,code1,language-python,lang-python,python,highlighted},language={python}]{\mdToken{Identifier,Python}{int}}% data-line={683}
{} and % data-line={683}
{}\mdCode[class={code,code1,language-python,lang-python,python,highlighted},language={python}]{\mdToken{Namespace,Identifier,Python}{SymbolicObject}}% data-line={683}
{}
and overrides a number of % data-line={684}
{}\mdCode[class={code,code1,language-python,lang-python,python,highlighted},language={python}]{\mdToken{Identifier,Python}{int}}% data-line={684}
{}{'}% data-line={684}
{}s arithmetic methods
(% data-line={685}
{}\mdCode[class={code,code1,language-python,lang-python,python,highlighted},language={python}]{\mdToken{Predefined,Python}{\_\_add\_\_}}% data-line={685}
{}, % data-line={685}
{}\mdCode[class={code,code1,language-python,lang-python,python,highlighted},language={python}]{\mdToken{Predefined,Python}{\_\_sub\_\_}}% data-line={685}
{}, % data-line={685}
{}\mdCode[class={code,code1,language-python,lang-python,python,highlighted},language={python}]{\mdToken{Predefined,Python}{\_\_mul\_\_}}% data-line={685}
{}, % data-line={685}
{}\mdCode[class={code,code1,language-python,lang-python,python,highlighted},language={python}]{\mdToken{Predefined,Python}{\_\_mod\_\_}}% data-line={685}
{}, % data-line={685}
{}\mdCode[class={code,code1,language-python,lang-python,python,highlighted},language={python}]{\mdToken{Predefined,Python}{\_\_floordiv\_}}% data-line={685}
{})
and bitwise methods
(% data-line={687}
{}\mdCode[class={code,code1,language-python,lang-python,python,highlighted},language={python}]{\mdToken{Predefined,Python}{\_\_and\_\_}}% data-line={687}
{}, % data-line={687}
{}\mdCode[class={code,code1,language-python,lang-python,python,highlighted},language={python}]{\mdToken{Predefined,Python}{\_\_or\_\_}}% data-line={687}
{}, % data-line={687}
{}\mdCode[class={code,code1,language-python,lang-python,python,highlighted},language={python}]{\mdToken{Predefined,Python}{\_\_xor\_\_}}% data-line={687}
{}, % data-line={687}
{}\mdCode[class={code,code1,language-python,lang-python,python,highlighted},language={python}]{\mdToken{Predefined,Python}{\_\_lshift\_\_}}% data-line={687}
{}, % data-line={687}
{}\mdCode[class={code,code1,language-python,lang-python,python,highlighted},language={python}]{\mdToken{Predefined,Python}{\_\_rshift\_\_}}% data-line={687}
{}).%
\end{mdP}%
\mdHxxx[id=sec-tracing-control-flow,label={[4.3]\{.heading-label\}},toc={},data-line={690},caption={[[4.3]\{.heading-label\}.{\enspace}]\{.heading-before\}Tracing control-flow},bookmark={4.3.{\enspace}Tracing control-flow}]{% data-line={690}
{}\mdSpan[class={heading-before}]{\mdSpan[class={heading-label}]{4.3}.{\enspace}}% data-line={690}
{}Tracing control-flow}\begin{mdP}[data-line={692}]%
% data-line={692}
{}As Python interprets a program, it will evaluate expressions, 
substituting the value of a variable in its place in an 
expression, applying operators (methods) to 
parameter values and assigning the return values of methods
to variables. Value of type % data-line={696}
{}\mdCode[class={code,code1,language-python,lang-python,python,highlighted},language={python}]{\mdToken{Namespace,Identifier,Python}{SymbolicInteger}}% data-line={696}
{} will simply flow
through this interpretation, without necessitating any change 
to the program or the interpreter. This takes care
of the case of the strongest-postcondition rule for assignment,
as elaborated in Section% data-line={700}
{}{\mdNbsp}\mdA[class={localref},target-element={h2}]{sp-refined}{}{\mdSpan[class={heading-label}]{3.2}}% data-line={700}
{}.%
\end{mdP}%
\begin{mdP}[class={indent},data-line={702}]%
% data-line={702}
{}The strong-postcondition rule for a conditional test requires
a little more work. In Python, any object can be tested in
an % data-line={704}
{}\mdCode[class={code,code1,language-python,lang-python,python,highlighted},language={python}]{\mdToken{Keyword,Python}{if}}% data-line={704}
{} or % data-line={704}
{}\mdCode[class={code,code1,language-python,lang-python,python,highlighted},language={python}]{\mdToken{Keyword,Python}{while}}% data-line={704}
{} condition or as the operand of a Boolean operation
(% data-line={705}
{}\mdCode[class={code,code1,language-python,lang-python,python,highlighted},language={python}]{\mdToken{Keyword,Python}{and}}% data-line={705}
{}, % data-line={705}
{}\mdCode[class={code,code1,language-python,lang-python,python,highlighted},language={python}]{\mdToken{Keyword,Python}{or}}% data-line={705}
{}, % data-line={705}
{}\mdCode[class={code,code1,language-python,lang-python,python,highlighted},language={python}]{\mdToken{Keyword,Python}{not}}% data-line={705}
{})
The Python base class % data-line={706}
{}\mdCode[class={code,code1,language-python,lang-python,python,highlighted},language={python}]{\mdToken{Identifier,Python}{object}}% data-line={706}
{} provides a method named % data-line={706}
{}\mdCode[class={code,code1,language-python,lang-python,python,highlighted},language={python}]{\mdToken{Predefined,Python}{\_\_bool\_\_}}% data-line={706}
{} that
the Python runtime calls whenever it needs to perform such a conditional test.
This hook provides us what we need to trace the conditional
control-flow of a Python execution.  We override this method in the class
% data-line={710}
{}\mdCode[class={code,code1,language-python,lang-python,python,highlighted},language={python}]{\mdToken{Namespace,Identifier,Python}{SymbolicObject}}% data-line={710}
{} in order to inform the % data-line={710}
{}\mdCode[class={code,code1,language-python,lang-python,python,highlighted},language={python}]{\mdToken{Namespace,Identifier,Python}{PathToConstraint}}% data-line={710}
{} object (defined
later) of the symbolic expression for the conditional (as captured by
the % data-line={712}
{}\mdCode[class={code,code1,language-python,lang-python,python,highlighted},language={python}]{\mdToken{Namespace,Identifier,Python}{SymbolicInteger}}% data-line={712}
{} subclass).%
\end{mdP}%
\begin{mdP}[class={indent},data-line={714}]%
% data-line={714}
{}Note that the use of this hook in combination with the
tainted types will only trace those conditionals
in a Python execution whose values inherit from % data-line={716}
{}\mdCode[class={code,code1,language-python,lang-python,python,highlighted},language={python}]{\mdToken{Namespace,Identifier,Python}{SymbolicObject}}% data-line={716}
{}; 
by definition, % data-line={717}
{}{\textquotedblleft}untainted{\textquotedblright}% data-line={717}
{} conditionals do not depend on symbolic 
inputs so there is no value in adding them to the path-condition.%
\end{mdP}%
\mdHxxx[id=sec-recording-path-conditions,label={[4.4]\{.heading-label\}},toc={},data-line={720},caption={[[4.4]\{.heading-label\}.{\enspace}]\{.heading-before\}Recording path-conditions},bookmark={4.4.{\enspace}Recording path-conditions}]{% data-line={720}
{}\mdSpan[class={heading-before}]{\mdSpan[class={heading-label}]{4.4}.{\enspace}}% data-line={720}
{}Recording path-conditions}\begin{mdP}[data-line={722}]%
% data-line={722}
{}A % data-line={722}
{}\mdCode[class={code,code1,language-python,lang-python,python,highlighted},language={python}]{\mdToken{Namespace,Identifier,Python}{Predicate}}% data-line={722}
{} records a conditional (more precisely the symbolic expression
found in % data-line={723}
{}\mdCode[class={code,code1,language-python,lang-python,python,highlighted},language={python}]{\mdToken{Namespace,Identifier,Python}{SymbolicInteger}}% data-line={723}
{}) and
which way it evaluated in an execution.  A % data-line={724}
{}\mdCode[class={code,code1,language-python,lang-python,python,highlighted},language={python}]{\mdToken{Namespace,Identifier,Python}{Constraint}}% data-line={724}
{}
has a % data-line={725}
{}\mdCode[class={code,code1,language-python,lang-python,python,highlighted},language={python}]{\mdToken{Namespace,Identifier,Python}{Predicate}}% data-line={725}
{}, a parent % data-line={725}
{}\mdCode[class={code,code1,language-python,lang-python,python,highlighted},language={python}]{\mdToken{Namespace,Identifier,Python}{Constraint}}% data-line={725}
{} and a set
of % data-line={726}
{}\mdCode[class={code,code1,language-python,lang-python,python,highlighted},language={python}]{\mdToken{Namespace,Identifier,Python}{Constraint}}% data-line={726}
{} children. % data-line={726}
{}\mdCode[class={code,code1,language-python,lang-python,python,highlighted},language={python}]{\mdToken{Namespace,Identifier,Python}{Constraints}}% data-line={726}
{} form a tree, where
each path starting from the root of the tree represents
a path-condition. The tree represents all path-conditions that have
been explored so far.%
\end{mdP}%
\begin{mdP}[class={indent},data-line={731}]%
% data-line={731}
{}The class % data-line={731}
{}\mdCode[class={code,code1,language-python,lang-python,python,highlighted},language={python}]{\mdToken{Namespace,Identifier,Python}{PathToConstraint}}% data-line={731}
{} has a reference to the root of 
the tree of % data-line={732}
{}\mdCode[class={code,code1,language-python,lang-python,python,highlighted},language={python}]{\mdToken{Namespace,Identifier,Python}{Constraint}}% data-line={732}
{}s
and is responsible for installing a new % data-line={733}
{}\mdCode[class={code,code1,language-python,lang-python,python,highlighted},language={python}]{\mdToken{Namespace,Identifier,Python}{Constraint}}% data-line={733}
{} in the tree
when notified by the overridden % data-line={734}
{}\mdCode[class={code,code1,language-python,lang-python,python,highlighted},language={python}]{\mdToken{Predefined,Python}{\_\_bool\_\_}}% data-line={734}
{} method of % data-line={734}
{}\mdCode[class={code,code1,language-python,lang-python,python,highlighted},language={python}]{\mdToken{Namespace,Identifier,Python}{SymbolicObject}}% data-line={734}
{}.
% data-line={735}
{}\mdCode[class={code,code1,language-python,lang-python,python,highlighted},language={python}]{\mdToken{Namespace,Identifier,Python}{PathToConstraint}}% data-line={735}
{} also tracks whether or not the current execution
is following an existing path in the tree and grows the
tree as needed. In fact, it actually
tracks whether or not the current execution follows a particular
% data-line={739}
{}\mdEm{expected path}% data-line={739}
{} in the tree.%
\end{mdP}%
\begin{mdP}[class={indent},data-line={741}]%
% data-line={741}
{}The expected path is the result
of the % data-line={742}
{}\mdCode[class={code,code1,language-python,lang-python,python,highlighted},language={python}]{\mdToken{Namespace,Identifier,Python}{ExplorationEngine}}% data-line={742}
{} picking a constraint % data-line={742}
{}\mdSpan[class={math-inline},elem={math-inline}]{$c$}% data-line={742}
{} in the tree,
and asking the ATP if the path-condition consisting of the prefix
of predicates up to but not including % data-line={744}
{}\mdSpan[class={math-inline},elem={math-inline}]{$c$}% data-line={744}
{} in the tree, 
followed by the negation of % data-line={745}
{}\mdSpan[class={math-inline},elem={math-inline}]{$c$}% data-line={745}
{}{'}% data-line={745}
{}s predicate is satisfiable. If the 
ATP returns % data-line={746}
{}{\textquotedblleft}satisfiable{\textquotedblright}% data-line={746}
{} (with a new input % data-line={746}
{}\mdSpan[class={math-inline},elem={math-inline}]{$i$}% data-line={746}
{}), then the assumption
is that path-condition prefix is sound (that is, the execution of
the program on input % data-line={748}
{}\mdSpan[class={math-inline},elem={math-inline}]{$i$}% data-line={748}
{} will follow the prefix).%
\end{mdP}%
\begin{mdP}[class={indent},data-line={750}]%
% data-line={750}
{}However, it is possible 
for the path-condition to be unsound and for 
the executed path to diverge early 
from the expected path, due to the fact that not every operation
has a symbolic encoding.  The tool simply reports the divergence
and continues to process the execution as usual (as a diverging
path may lead to some other interesting part of the code).%
\end{mdP}%
\mdHxxx[id=sec-from-symbolic-types-to-z3,label={[4.5]\{.heading-label\}},toc={},data-line={758},caption={[[4.5]\{.heading-label\}.{\enspace}]\{.heading-before\}From symbolic types to Z3},bookmark={4.5.{\enspace}From symbolic types to Z3}]{% data-line={758}
{}\mdSpan[class={heading-before}]{\mdSpan[class={heading-label}]{4.5}.{\enspace}}% data-line={758}
{}From symbolic types to Z3}\begin{mdP}[data-line={760}]%
% data-line={760}
{}As we have explained DSE, the symbolic expressions are 
represented at the level of the source language. As detailed later
in Section% data-line={762}
{}{\mdNbsp}\mdA[class={localref},target-element={h1}]{sec-int2z3}{}{\mdSpan[class={heading-label}]{5}}% data-line={762}
{}, we must translate 
from the source language to the input language of an
automated theorem prover (ATP), in this case% data-line={764}
{}{\mdNbsp}\mdA{http://z3.codeplex.org/}{}{Z3}% data-line={764}
{}.  This
separation of languages is quite useful, as we may have
the need to translate a given symbolic expression
to the ATP% data-line={767}
{}{'}% data-line={767}
{}s language multiple times, to make use of different
features of the underlying ATP. 
Furthermore, this separation
of concerns allows us to easily retarget the DSE tool to a
different ATP.%
\end{mdP}%
\begin{mdP}[class={indent},data-line={773}]%
% data-line={773}
{}The base class % data-line={773}
{}\mdCode[class={code,code1,language-python,lang-python,python,highlighted},language={python}]{\mdToken{Namespace,Identifier,Python}{Z3Expression}}% data-line={773}
{} represents a Z3 formula. The two 
subclasses % data-line={774}
{}\mdCode[class={code,code1,language-python,lang-python,python,highlighted},language={python}]{\mdToken{Namespace,Identifier,Python}{Z3Integer}}% data-line={774}
{} and % data-line={774}
{}\mdCode[class={code,code1,language-python,lang-python,python,highlighted},language={python}]{\mdToken{Namespace,Identifier,Python}{Z3BitVector}}% data-line={774}
{} represent different ways 
to model arithmetic reasoning about integers in Z3. We will describe
the details of these encodings in Section% data-line={776}
{}{\mdNbsp}\mdA[class={localref},target-element={h1}]{sec-int2z3}{}{\mdSpan[class={heading-label}]{5}}% data-line={776}
{}.%
\end{mdP}%
\begin{mdP}[class={indent},data-line={778}]%
% data-line={778}
{}The class % data-line={778}
{}\mdCode[class={code,code1,language-python,lang-python,python,highlighted},language={python}]{\mdToken{Namespace,Identifier,Python}{Z3Wrapper}}% data-line={778}
{} is responsible for performing the
translation from the source language (Python) to Z3% data-line={779}
{}{'}% data-line={779}
{}s input language, 
invoking Z3, and lifting a Z3 answer back to the level of Python. 
The % data-line={781}
{}\mdCode[class={code,code1,language-python,lang-python,python,highlighted},language={python}]{\mdToken{Identifier,Python}{findCounterexample}}% data-line={781}
{} method does all the work, taking as
input a list of % data-line={782}
{}\mdCode[class={code,code1,language-python,lang-python,python,highlighted},language={python}]{\mdToken{Namespace,Identifier,Python}{Predicate}}% data-line={782}
{}s (called % data-line={782}
{}\mdCode[class={code,code1,language-python,lang-python,python,highlighted},language={python}]{\mdToken{Identifier,Python}{assertions}}% data-line={782}
{}) 
as well as a single % data-line={783}
{}\mdCode[class={code,code1,language-python,lang-python,python,highlighted},language={python}]{\mdToken{Namespace,Identifier,Python}{Predicate}}% data-line={783}
{} (called
the % data-line={784}
{}\mdCode[class={code,code1,language-python,lang-python,python,highlighted},language={python}]{\mdToken{Identifier,Python}{query}}% data-line={784}
{}). The % data-line={784}
{}\mdCode[class={code,code1,language-python,lang-python,python,highlighted},language={python}]{\mdToken{Identifier,Python}{assertions}}% data-line={784}
{} represent a path-condition
prefix derived from the % data-line={785}
{}\mdCode[class={code,code1,language-python,lang-python,python,highlighted},language={python}]{\mdToken{Namespace,Identifier,Python}{Constraint}}% data-line={785}
{} tree that we wish the next
execution to follow, while % data-line={786}
{}\mdCode[class={code,code1,language-python,lang-python,python,highlighted},language={python}]{\mdToken{Identifier,Python}{query}}% data-line={786}
{} represents the predicate
following the prefix in the tree that we will negate.%
\end{mdP}%
\begin{mdP}[class={indent,para-continue},data-line={789}]%
% data-line={789}
{}The method constructs the formula%
\end{mdP}%
\begin{mdDiv}[class={equation,align-center,para-block},label={[(4)]\{.equation-label\}},elem={equation},line-adjust={0},data-line={791}]%
% data-line={791}
{}\mdSpan[class={equation-before}]{\mdSpan[class={equation-label}]{(4)}}% data-line={791}
{}
\begin{mdDiv}[class={math,para-block,input-math},elem={math},color={},math-needpdf={},line-adjust={0},data-line={792}]%
\begin{mdDiv}[class={math-display}]%
\[% data-line={792}
(\bigwedge_{a \in asserts} a) \wedge \neg query\]%
\end{mdDiv}%%
\end{mdDiv}%%
\end{mdDiv}%
\begin{mdP}[data-line={795}]%
% data-line={795}
{}and asks Z3 if it is satisfiable. The method performs
a standard syntactic % data-line={796}
{}{\textquotedblleft}cone of influence{\textquotedblright}% data-line={796}
{} (CIF)
reduction on the % data-line={797}
{}\mdCode[class={code,code1,language-python,lang-python,python,highlighted},language={python}]{\mdToken{Identifier,Python}{asserts}}% data-line={797}
{} with respect to the 
% data-line={798}
{}\mdCode[class={code,code1,language-python,lang-python,python,highlighted},language={python}]{\mdToken{Identifier,Python}{query}}% data-line={798}
{} to shrink the size of the formula. For example,
if % data-line={799}
{}\mdCode[class={code,code1,language-python,lang-python,python,highlighted},language={python}]{\mdToken{Identifier,Python}{asserts}}% data-line={799}
{} is the set of predicates % data-line={799}
{}\mdSpan[class={math-inline},elem={math-inline}]{$\{ (x<a), (a<0), (y>0) \}$}% data-line={799}
{} and
the query is % data-line={800}
{}\mdSpan[class={math-inline},elem={math-inline}]{$(x=0)$}% data-line={800}
{}, then the CIF yields the
set  % data-line={801}
{}\mdSpan[class={math-inline},elem={math-inline}]{$\{ (x<a), (a<0) \}$}% data-line={801}
{}, which does not include the predicate % data-line={801}
{}\mdSpan[class={math-inline},elem={math-inline}]{$(y>0)$}% data-line={801}
{},
as the variable % data-line={802}
{}\mdSpan[class={math-inline},elem={math-inline}]{$y$}% data-line={802}
{} is not in the set of variables (transitively)
related to variable % data-line={803}
{}\mdSpan[class={math-inline},elem={math-inline}]{$x$}% data-line={803}
{}.%
\end{mdP}%
\begin{mdP}[class={indent},data-line={805}]%
% data-line={805}
{}If the formula is satisfiable a model is requested
from Z3 and lifted back to Python% data-line={806}
{}{'}% data-line={806}
{}s type universe.  Note that
because of the CIF reduction, the model may not mention certain
input variables, in which case we simply keep their values from
the execution from which the % data-line={809}
{}\mdCode[class={code,code1,language-python,lang-python,python,highlighted},language={python}]{\mdToken{Identifier,Python}{asserts}}% data-line={809}
{} and % data-line={809}
{}\mdCode[class={code,code1,language-python,lang-python,python,highlighted},language={python}]{\mdToken{Identifier,Python}{query}}% data-line={809}
{} were derived.%
\end{mdP}%
\mdHxxx[id=sec-putting-it-all-together,label={[4.6]\{.heading-label\}},toc={},data-line={811},caption={[[4.6]\{.heading-label\}.{\enspace}]\{.heading-before\}Putting it all together},bookmark={4.6.{\enspace}Putting it all together}]{% data-line={811}
{}\mdSpan[class={heading-before}]{\mdSpan[class={heading-label}]{4.6}.{\enspace}}% data-line={811}
{}Putting it all together}\begin{mdP}[data-line={813}]%
% data-line={813}
{}The class % data-line={813}
{}\mdCode[class={code,code1,language-python,lang-python,python,highlighted},language={python}]{\mdToken{Namespace,Identifier,Python}{ExplorationEngine}}% data-line={813}
{} ties everything together. It kicks off
an execution of the Python code under test using % data-line={814}
{}\mdCode[class={code,code1,language-python,lang-python,python,highlighted},language={python}]{\mdToken{Namespace,Identifier,Python}{FunctionInvocation}}% data-line={814}
{}.
As the Python code executes, building symbolic expressions via % data-line={815}
{}\mdCode[class={code,code1,language-python,lang-python,python,highlighted},language={python}]{\mdToken{Namespace,Identifier,Python}{SymbolicType}}% data-line={815}
{}
and its subclasses, callbacks to % data-line={816}
{}\mdCode[class={code,code1,language-python,lang-python,python,highlighted},language={python}]{\mdToken{Namespace,Identifier,Python}{PathToConstraint}}% data-line={816}
{} create
a path-condition, represented by % data-line={817}
{}\mdCode[class={code,code1,language-python,lang-python,python,highlighted},language={python}]{\mdToken{Namespace,Identifier,Python}{Constraint}}% data-line={817}
{} and % data-line={817}
{}\mdCode[class={code,code1,language-python,lang-python,python,highlighted},language={python}]{\mdToken{Namespace,Identifier,Python}{Predicate}}% data-line={817}
{}. 
Newly discovered % data-line={818}
{}\mdCode[class={code,code1,language-python,lang-python,python,highlighted},language={python}]{\mdToken{Namespace,Identifier,Python}{Constraints}}% data-line={818}
{} are added to the end of a deque maintained by
% data-line={819}
{}\mdCode[class={code,code1,language-python,lang-python,python,highlighted},language={python}]{\mdToken{Namespace,Identifier,Python}{ExplorationEngine}}% data-line={819}
{}.%
\end{mdP}%
\begin{mdP}[class={indent},data-line={821}]%
% data-line={821}
{}Given the first seed execution, % data-line={821}
{}\mdCode[class={code,code1,language-python,lang-python,python,highlighted},language={python}]{\mdToken{Namespace,Identifier,Python}{ExplorationEngine}}% data-line={821}
{} starts the work of 
exploring paths in a breadth-first fashion. It removes a % data-line={822}
{}\mdCode[class={code,code1,language-python,lang-python,python,highlighted},language={python}]{\mdToken{Namespace,Identifier,Python}{Constraint}}% data-line={822}
{} % data-line={822}
{}\mdSpan[class={math-inline},elem={math-inline}]{$c$}% data-line={822}
{}
from the front of its deque and, if % data-line={823}
{}\mdSpan[class={math-inline},elem={math-inline}]{$c$}% data-line={823}
{} has not been already % data-line={823}
{}{\textquotedblleft}processed{\textquotedblright}% data-line={823}
{}, 
uses % data-line={824}
{}\mdCode[class={code,code1,language-python,lang-python,python,highlighted},language={python}]{\mdToken{Namespace,Identifier,Python}{Z3Wrapper}}% data-line={824}
{} to find a new input (as discussed in the previous section)
where % data-line={825}
{}\mdSpan[class={math-inline},elem={math-inline}]{$c$}% data-line={825}
{} is the query (to be negated) and the path to % data-line={825}
{}\mdSpan[class={math-inline},elem={math-inline}]{$c$}% data-line={825}
{} in the 
% data-line={826}
{}\mdCode[class={code,code1,language-python,lang-python,python,highlighted},language={python}]{\mdToken{Namespace,Identifier,Python}{Constraint}}% data-line={826}
{} tree forms the assertions.%
\end{mdP}%
\begin{mdP}[class={indent},data-line={828}]%
% data-line={828}
{}A % data-line={828}
{}\mdCode[class={code,code1,language-python,lang-python,python,highlighted},language={python}]{\mdToken{Namespace,Identifier,Python}{Constraint}}% data-line={828}
{} % data-line={828}
{}\mdSpan[class={math-inline},elem={math-inline}]{$c$}% data-line={828}
{} in the tree is considered % data-line={828}
{}{\textquotedblleft}processed{\textquotedblright}% data-line={828}
{} if an execution 
has covered % data-line={829}
{}\mdSpan[class={math-inline},elem={math-inline}]{$c'$}% data-line={829}
{}, a sibling of % data-line={829}
{}\mdSpan[class={math-inline},elem={math-inline}]{$c$}% data-line={829}
{} in the tree that represents
the negation of the predicate associated with % data-line={830}
{}\mdSpan[class={math-inline},elem={math-inline}]{$c$}% data-line={830}
{}, or if constraint % data-line={830}
{}\mdSpan[class={math-inline},elem={math-inline}]{$c$}% data-line={830}
{}
has been removed from the deque.%
\end{mdP}%
\mdHxx[id=sec-int2z3,label={[5]\{.heading-label\}},toc={},data-line={833},caption={[[5]\{.heading-label\}.{\enspace}]\{.heading-before\}From Python Integers to Z3 Arithmetic},bookmark={5.{\enspace}From Python Integers to Z3 Arithmetic}]{% data-line={833}
{}\mdSpan[class={heading-before}]{\mdSpan[class={heading-label}]{5}.{\enspace}}% data-line={833}
{}From Python Integers to Z3 Arithmetic}\begin{mdP}[data-line={835}]%
% data-line={835}
{}In languages such as C and Java, integers are finite-precision,
generally limited to the size of a machine word (32 or 64 bits, for example).
For such languages, satisfiability of finite-precision integer arithmetic
is decidable and can be reduced to Z3% data-line={838}
{}{'}% data-line={838}
{}s theory of bit-vectors, where
each arithmetic operation is encoded by a circuit. This translation permits 
reasoning about non-linear arithmetic problems, such as 
% data-line={841}
{}\mdSpan[class={math-inline},elem={math-inline}]{$\exists x,y,z : x*z + y \leq (z/y)+5$}% data-line={841}
{}.%
\end{mdP}%
\begin{mdP}[class={indent},data-line={843}]%
% data-line={843}
{}Python (3.0) integers, however, are not finite-precision. They are only
limited by the size of machine memory. This means, for example, that
Python integers don% data-line={845}
{}{'}% data-line={845}
{}t overflow or underflow. It also means that
we can% data-line={846}
{}{'}% data-line={846}
{}t hope to decide algorithmically whether or not a given
equation over integer variables has a solution in general. Hilbert% data-line={847}
{}{'}% data-line={847}
{}s famous
10th problem and its solution by Matiyasevich tells us that it is
undecidable whether or not a polynomial
equation of the form % data-line={850}
{}\mdSpan[class={math-inline},elem={math-inline}]{$p(x_1, \ldots, x_n) = 0$}% data-line={850}
{} with integer coefficients
has an solution in the integers.%
\end{mdP}%
\begin{mdP}[class={indent},data-line={853}]%
% data-line={853}
{}This means that we will resort to heuristic approaches in our use
of the% data-line={854}
{}{\mdNbsp}\mdA{http://z3.codeplex.org/}{}{Z3}% data-line={854}
{} ATP.  The special case of linear integer arithmetic (LIA)
is decidable and supported by Z3. In order to deal with non-linear operations,
we use uninterpreted functions (UF). Thus, if Z3 returns
% data-line={857}
{}{\textquotedblleft}unsatisfiable{\textquotedblright}% data-line={857}
{} we know that there is no solution, but if the
Z3 % data-line={858}
{}{\textquotedblleft}satisfiable{\textquotedblright}% data-line={858}
{}, we must treat the answer as a % data-line={858}
{}{\textquotedblleft}don{'}t know{\textquotedblright}% data-line={858}
{}. 
The class % data-line={859}
{}\mdCode[class={code,code1,language-python,lang-python,python,highlighted},language={python}]{\mdToken{Namespace,Identifier,Python}{Z3Integer}}% data-line={859}
{} is used to translate a symbolic expression
into the theory LIA+UF and check for unsatisfiability. We leave it as an
implementation exercise to check if a symbolic expression can
be converted to LIA (without the use of UF) in order to make
use of % data-line={863}
{}{\textquotedblleft}satisfiable{\textquotedblright}% data-line={863}
{} answers from the LIA solver.%
\end{mdP}%
\begin{mdP}[class={indent},data-line={865}]%
% data-line={865}
{}If the translation to % data-line={865}
{}\mdCode[class={code,code1,language-python,lang-python,python,highlighted},language={python}]{\mdToken{Namespace,Identifier,Python}{Z3Integer}}% data-line={865}
{} does not return % data-line={865}
{}{\textquotedblleft}unsatisfiable{\textquotedblright}% data-line={865}
{}, we
use Z3% data-line={866}
{}{'}% data-line={866}
{}s bit vector decision procedure (via the class
% data-line={867}
{}\mdCode[class={code,code1,language-python,lang-python,python,highlighted},language={python}]{\mdToken{Namespace,Identifier,Python}{Z3BitVector}}% data-line={867}
{}) to heuristically
try to find satisfiable answers, even in the presence of non-linear 
arithmetic. We start with bitvectors of size % data-line={869}
{}\mdSpan[class={math-inline},elem={math-inline}]{$N=32$}% data-line={869}
{} and % data-line={869}
{}\mdEm{bound}% data-line={869}
{} the values
of the symbolic constants to fit within 8 bits in order to find 
satisfiable solutions
with small values. Also, because Python integers do not overflow/underflow, 
the bound helps us reserve space in the bitvector to allow the
results of operations to exceed the bound while not overflowing
the bitvector. As long as Z3 returns % data-line={875}
{}{\textquotedblleft}unsatisfiable{\textquotedblright}% data-line={875}
{} we increase
the bound. If the bound reaches % data-line={876}
{}\mdSpan[class={math-inline},elem={math-inline}]{$N$}% data-line={876}
{}, we increase % data-line={876}
{}\mdSpan[class={math-inline},elem={math-inline}]{$N$}% data-line={876}
{} by 8 bits,
leaving the bound where it is and continue.%
\end{mdP}%
\begin{mdP}[class={indent},data-line={879}]%
% data-line={879}
{}If Z3 returns
% data-line={880}
{}{\textquotedblleft}satisfiable{\textquotedblright}% data-line={880}
{}, it may be the case that Z3 found a solution
that involved overflow in the bitvector world of arithmetic
(modulo % data-line={882}
{}\mdSpan[class={math-inline},elem={math-inline}]{$2^N-1$}% data-line={882}
{}). Therefore,
the solution is validated back in the
Python world by evaluating the formula under that solution
using Python semantics. 
If the formula does not evaluate to the same
value in both worlds, then we increase % data-line={887}
{}\mdSpan[class={math-inline},elem={math-inline}]{$N$}% data-line={887}
{} by 8 bits (to 
create a gap between the bound and % data-line={888}
{}\mdSpan[class={math-inline},elem={math-inline}]{$N$}% data-line={888}
{}) and continue to search
for a solution.%
\end{mdP}%
\begin{mdP}[class={indent},data-line={891}]%
% data-line={891}
{}The process terminates when we find a valid satisfying solution 
or % data-line={892}
{}\mdSpan[class={math-inline},elem={math-inline}]{$N=64$}% data-line={892}
{} and the bound reaches 64 (in which case, we return % data-line={892}
{}{\textquotedblleft}don{'}t know{\textquotedblright}% data-line={892}
{}).%
\end{mdP}%
\mdHxx[id=sec-extensions,label={[6]\{.heading-label\}},toc={},data-line={894},caption={[[6]\{.heading-label\}.{\enspace}]\{.heading-before\}Extensions},bookmark={6.{\enspace}Extensions}]{% data-line={894}
{}\mdSpan[class={heading-before}]{\mdSpan[class={heading-label}]{6}.{\enspace}}% data-line={894}
{}Extensions}\begin{mdP}[data-line={896}]%
% data-line={896}
{}We have presented the basics of dynamic symbolic execution 
(for Python).
A more thorough treatment would deal with other data types besides
integers, such as Python dictionaries, strings and lists, each
of which presents their own challenges for symbolic reasoning. 
There are many other interesting challenges in DSE, such
as dealing with user-defined classes (rather than built-in types
as done here) and multi-threaded execution.%
\end{mdP}%
\mdHxx[id=sec-acknowledgements,label={[7]\{.heading-label\}},toc={},data-line={905},caption={[[7]\{.heading-label\}.{\enspace}]\{.heading-before\}Acknowledgements},bookmark={7.{\enspace}Acknowledgements}]{% data-line={905}
{}\mdSpan[class={heading-before}]{\mdSpan[class={heading-label}]{7}.{\enspace}}% data-line={905}
{}Acknowledgements}\begin{mdP}[data-line={907}]%
% data-line={907}
{}Many thanks to the students of the 2014 Marktoberdorf Summer School
on Dependable Software Systems Engineering
for their questions and feedback about the first author% data-line={909}
{}{'}% data-line={909}
{}s lectures on dynamic
symbolic execution. The following students of the summer school
helpfully provided tests for the% data-line={911}
{}{\mdNbsp}\mdA{https://github.com/thomasjball/PyExZ3/}{}{PyExZ3}% data-line={911}
{} tool: Daniel Darvas,
Damien Rusinek, Christian Dehnert and Thomas Pani. Thanks also to Peter
Chapman for his contributions.%
\end{mdP}%
\mdHxx[id=sec-references,data-line={971},caption={References}]{% data-line={971}
{}References}\begin{mdDiv}[class={tex,input-tex},elem={tex},data-line={972;out{\textbackslash}DSE-bib.bbl:1}]%
\begin{mdBibliography}[class={bibliography,bib-numeric},elem={bibliography},bibstyle={plainnat},bibdata={dse},caption={14},data-line={972;out{\textbackslash}DSE-bib.bbl:2}]%
\begin{mdBibitem}[class={bibitem},id=cadare05,label={[1]\{.bibitem-label\}},elem={bibitem},cite-label={Cadar and Engler(2005)},caption={Cristian Cadar and Dawson{\mdNbsp}R. Engler. \\Execution generated test cases: How to make systems code crash itself. \\In \_Proceedings of 12th International SPIN Workshop\_, pages 2--23, 2005.},searchterm={Cristian+Cadar+Dawson+Engler+Execution+generated+test+cases+make+systems+code+crash+itself+\_Proceedings+International+SPIN+Workshop\_+pages++},data-line={972;out{\textbackslash}DSE-bib.bbl:5}]%
% data-line={972;out\DSE-bib.bbl:6}
{}\mdSpan[class={bibitem-before}]{[\mdSpan[class={bibitem-label}]{1}]{\mdNbsp}{\mdNbsp}}% data-line={972;out\DSE-bib.bbl:6}
{}Cristian Cadar and Dawson% data-line={972;out\DSE-bib.bbl:6}
{}{\mdNbsp}% data-line={972;out\DSE-bib.bbl:6}
{}R. Engler.
% data-line={972;out\DSE-bib.bbl:7}
{}\mdSpan[class={newblock}]{}% data-line={972;out\DSE-bib.bbl:7}
{} Execution generated test cases: How to make systems code crash
  itself.
% data-line={972;out\DSE-bib.bbl:9}
{}\mdSpan[class={newblock}]{}% data-line={972;out\DSE-bib.bbl:9}
{} In % data-line={972;out\DSE-bib.bbl:9}
{}\mdEm{Proceedings of 12th International SPIN Workshop}% data-line={972;out\DSE-bib.bbl:9}
{}, pages
  2% data-line={972;out\DSE-bib.bbl:10}
{}{\textendash}% data-line={972;out\DSE-bib.bbl:10}
{}23, 2005.%
\end{mdBibitem}%
\begin{mdBibitem}[class={bibitem},id=cadars13,label={[2]\{.bibitem-label\}},elem={bibitem},cite-label={Cadar and Sen(2013)},caption={Cristian Cadar and Koushik Sen. \\Symbolic execution for software testing: three decades later.},searchterm={Symbolic+execution+software+testing+three+decades+later++Cristian+Cadar+Koushik+},data-line={972;out{\textbackslash}DSE-bib.bbl:13}]%
% data-line={972;out\DSE-bib.bbl:14}
{}\mdSpan[class={bibitem-before}]{[\mdSpan[class={bibitem-label}]{2}]{\mdNbsp}{\mdNbsp}}% data-line={972;out\DSE-bib.bbl:14}
{}Cristian Cadar and Koushik Sen.
% data-line={972;out\DSE-bib.bbl:15}
{}\mdSpan[class={newblock}]{}% data-line={972;out\DSE-bib.bbl:15}
{} Symbolic execution for software testing: three decades later.
% data-line={972;out\DSE-bib.bbl:16}
{}\mdSpan[class={newblock}]{}% data-line={972;out\DSE-bib.bbl:16}
{} % data-line={972;out\DSE-bib.bbl:16}
{}\mdEm{Communications of the ACM}% data-line={972;out\DSE-bib.bbl:16}
{}, 56% data-line={972;out\DSE-bib.bbl:16}
{}\mdSpan[penalty={0}]{}% data-line={972;out\DSE-bib.bbl:16}
{} (2):% data-line={972;out\DSE-bib.bbl:16}
{}\mdSpan[penalty={0}]{}% data-line={972;out\DSE-bib.bbl:16}
{} 82% data-line={972;out\DSE-bib.bbl:16}
{}{\textendash}% data-line={972;out\DSE-bib.bbl:16}
{}90,
  2013.%
\end{mdBibitem}%
\begin{mdBibitem}[class={bibitem},id=cadargpde06,label={[3]\{.bibitem-label\}},elem={bibitem},cite-label={Cadar et{\mdNbsp}al.(2006)Cadar, Ganesh, Pawlowski, Dill, and   Engler},caption={Cristian Cadar, Vijay Ganesh, Peter{\mdNbsp}M. Pawlowski, David{\mdNbsp}L. Dill, and Dawson{\mdNbsp}R. Engler. \\EXE: automatically generating inputs of death. \\In \_Proceedings of the 13th ACM Conference on Computer and Communications Security\_, pages 322--335, 2006.},searchterm={Cristian+Cadar+Vijay+Ganesh+Peter+Pawlowski+David+Dill+Dawson+Engler+automatically+generating+inputs+death+\_Proceedings+Conference+Computer+Communications+Security\_+pages++},data-line={972;out{\textbackslash}DSE-bib.bbl:20}]%
% data-line={972;out\DSE-bib.bbl:21}
{}\mdSpan[class={bibitem-before}]{[\mdSpan[class={bibitem-label}]{3}]{\mdNbsp}{\mdNbsp}}% data-line={972;out\DSE-bib.bbl:21}
{}Cristian Cadar, Vijay Ganesh, Peter% data-line={972;out\DSE-bib.bbl:21}
{}{\mdNbsp}% data-line={972;out\DSE-bib.bbl:21}
{}M. Pawlowski, David% data-line={972;out\DSE-bib.bbl:21}
{}{\mdNbsp}% data-line={972;out\DSE-bib.bbl:21}
{}L. Dill, and Dawson% data-line={972;out\DSE-bib.bbl:21}
{}{\mdNbsp}% data-line={972;out\DSE-bib.bbl:21}
{}R.
  Engler.
% data-line={972;out\DSE-bib.bbl:23}
{}\mdSpan[class={newblock}]{}% data-line={972;out\DSE-bib.bbl:23}
{} EXE: automatically generating inputs of death.
% data-line={972;out\DSE-bib.bbl:24}
{}\mdSpan[class={newblock}]{}% data-line={972;out\DSE-bib.bbl:24}
{} In % data-line={972;out\DSE-bib.bbl:24}
{}\mdEm{Proceedings of the 13th ACM Conference on Computer and
  Communications Security}% data-line={972;out\DSE-bib.bbl:25}
{}, pages 322% data-line={972;out\DSE-bib.bbl:25}
{}{\textendash}% data-line={972;out\DSE-bib.bbl:25}
{}335, 2006.%
\end{mdBibitem}%
\begin{mdBibitem}[class={bibitem},id=clarke76,label={[4]\{.bibitem-label\}},elem={bibitem},cite-label={Clarke(1976)},caption={Lori{\mdNbsp}A. Clarke. \\A system to generate test data and symbolically execute programs.},searchterm={+system+generate+test+data+symbolically+execute+programs++Lori+Clarke+},data-line={972;out{\textbackslash}DSE-bib.bbl:28}]%
% data-line={972;out\DSE-bib.bbl:29}
{}\mdSpan[class={bibitem-before}]{[\mdSpan[class={bibitem-label}]{4}]{\mdNbsp}{\mdNbsp}}% data-line={972;out\DSE-bib.bbl:29}
{}Lori% data-line={972;out\DSE-bib.bbl:29}
{}{\mdNbsp}% data-line={972;out\DSE-bib.bbl:29}
{}A. Clarke.
% data-line={972;out\DSE-bib.bbl:30}
{}\mdSpan[class={newblock}]{}% data-line={972;out\DSE-bib.bbl:30}
{} A system to generate test data and symbolically execute programs.
% data-line={972;out\DSE-bib.bbl:31}
{}\mdSpan[class={newblock}]{}% data-line={972;out\DSE-bib.bbl:31}
{} % data-line={972;out\DSE-bib.bbl:31}
{}\mdEm{IEEE Transactions on Software Engineering}% data-line={972;out\DSE-bib.bbl:31}
{}, 2% data-line={972;out\DSE-bib.bbl:31}
{}\mdSpan[penalty={0}]{}% data-line={972;out\DSE-bib.bbl:31}
{}
  (3):% data-line={972;out\DSE-bib.bbl:32}
{}\mdSpan[penalty={0}]{}% data-line={972;out\DSE-bib.bbl:32}
{} 215% data-line={972;out\DSE-bib.bbl:32}
{}{\textendash}% data-line={972;out\DSE-bib.bbl:32}
{}222, 1976.%
\end{mdBibitem}%
\begin{mdBibitem}[class={bibitem},id=demourab08,label={[5]\{.bibitem-label\}},elem={bibitem},cite-label={de{\mdNbsp}Moura and Bj{\o}rner(2008)},caption={Leonardo{\mdNbsp}Mendon{\c{c}}a de{\mdNbsp}Moura and Nikolaj Bj{\o}rner. \\Z3: an efficient SMT solver.},searchterm={+efficient+solver++Leonardo+Mendon+Moura+Nikolaj+rner+},data-line={972;out{\textbackslash}DSE-bib.bbl:35}]%
% data-line={972;out\DSE-bib.bbl:36}
{}\mdSpan[class={bibitem-before}]{[\mdSpan[class={bibitem-label}]{5}]{\mdNbsp}{\mdNbsp}}% data-line={972;out\DSE-bib.bbl:36}
{}Leonardo% data-line={972;out\DSE-bib.bbl:36}
{}{\mdNbsp}% data-line={972;out\DSE-bib.bbl:36}
{}Mendon% data-line={972;out\DSE-bib.bbl:36}
{}{\c{c}}% data-line={972;out\DSE-bib.bbl:36}
{}a de% data-line={972;out\DSE-bib.bbl:36}
{}{\mdNbsp}% data-line={972;out\DSE-bib.bbl:36}
{}Moura and Nikolaj Bj% data-line={972;out\DSE-bib.bbl:36}
{}{\o}% data-line={972;out\DSE-bib.bbl:36}
{}rner.
% data-line={972;out\DSE-bib.bbl:37}
{}\mdSpan[class={newblock}]{}% data-line={972;out\DSE-bib.bbl:37}
{} Z3: an efficient SMT solver.
% data-line={972;out\DSE-bib.bbl:38}
{}\mdSpan[class={newblock}]{}% data-line={972;out\DSE-bib.bbl:38}
{} In % data-line={972;out\DSE-bib.bbl:38}
{}\mdEm{Proceedings of the 14th International Conference of Tools
  and Algorithms for the Construction and Analysis of Systems}% data-line={972;out\DSE-bib.bbl:39}
{}, pages 337% data-line={972;out\DSE-bib.bbl:39}
{}{\textendash}% data-line={972;out\DSE-bib.bbl:39}
{}340,
  2008.%
\end{mdBibitem}%
\begin{mdBibitem}[class={bibitem},id=dijkstra76,label={[6]\{.bibitem-label\}},elem={bibitem},cite-label={Dijkstra(1976)},caption={Edsger{\mdNbsp}W. Dijkstra. \\\_A Discipline of Programming\_.},searchterm={+Discipline+Programming\_++Edsger+Dijkstra+},data-line={972;out{\textbackslash}DSE-bib.bbl:43}]%
% data-line={972;out\DSE-bib.bbl:44}
{}\mdSpan[class={bibitem-before}]{[\mdSpan[class={bibitem-label}]{6}]{\mdNbsp}{\mdNbsp}}% data-line={972;out\DSE-bib.bbl:44}
{}Edsger% data-line={972;out\DSE-bib.bbl:44}
{}{\mdNbsp}% data-line={972;out\DSE-bib.bbl:44}
{}W. Dijkstra.
% data-line={972;out\DSE-bib.bbl:45}
{}\mdSpan[class={newblock}]{}% data-line={972;out\DSE-bib.bbl:45}
{} % data-line={972;out\DSE-bib.bbl:45}
{}\mdEm{A Discipline of Programming}% data-line={972;out\DSE-bib.bbl:45}
{}.
% data-line={972;out\DSE-bib.bbl:46}
{}\mdSpan[class={newblock}]{}% data-line={972;out\DSE-bib.bbl:46}
{} Prentice-Hall, 1976.%
\end{mdBibitem}%
\begin{mdBibitem}[class={bibitem},id=godefroid11,label={[7]\{.bibitem-label\}},elem={bibitem},cite-label={Godefroid(2011)},caption={Patrice Godefroid. \\Higher-order test generation.},searchterm={Higher+order+test+generation++Patrice+Godefroid+},data-line={972;out{\textbackslash}DSE-bib.bbl:49}]%
% data-line={972;out\DSE-bib.bbl:50}
{}\mdSpan[class={bibitem-before}]{[\mdSpan[class={bibitem-label}]{7}]{\mdNbsp}{\mdNbsp}}% data-line={972;out\DSE-bib.bbl:50}
{}Patrice Godefroid.
% data-line={972;out\DSE-bib.bbl:51}
{}\mdSpan[class={newblock}]{}% data-line={972;out\DSE-bib.bbl:51}
{} Higher-order test generation.
% data-line={972;out\DSE-bib.bbl:52}
{}\mdSpan[class={newblock}]{}% data-line={972;out\DSE-bib.bbl:52}
{} In % data-line={972;out\DSE-bib.bbl:52}
{}\mdEm{Proceedings of the ACM SIGPLAN Conference on Programming
  Language Design and Implementation}% data-line={972;out\DSE-bib.bbl:53}
{}, pages 258% data-line={972;out\DSE-bib.bbl:53}
{}{\textendash}% data-line={972;out\DSE-bib.bbl:53}
{}269, 2011.%
\end{mdBibitem}%
\begin{mdBibitem}[class={bibitem},id=godefroidks05,label={[8]\{.bibitem-label\}},elem={bibitem},cite-label={Godefroid et{\mdNbsp}al.(2005)Godefroid, Klarlund, and Sen},caption={Patrice Godefroid, Nils Klarlund, and Koushik Sen. \\DART: directed automated random testing.},searchterm={DART+directed+automated+random+testing++Patrice+Godefroid+Nils+Klarlund+Koushik+},data-line={972;out{\textbackslash}DSE-bib.bbl:56}]%
% data-line={972;out\DSE-bib.bbl:57}
{}\mdSpan[class={bibitem-before}]{[\mdSpan[class={bibitem-label}]{8}]{\mdNbsp}{\mdNbsp}}% data-line={972;out\DSE-bib.bbl:57}
{}Patrice Godefroid, Nils Klarlund, and Koushik Sen.
% data-line={972;out\DSE-bib.bbl:58}
{}\mdSpan[class={newblock}]{}% data-line={972;out\DSE-bib.bbl:58}
{} DART: directed automated random testing.
% data-line={972;out\DSE-bib.bbl:59}
{}\mdSpan[class={newblock}]{}% data-line={972;out\DSE-bib.bbl:59}
{} In % data-line={972;out\DSE-bib.bbl:59}
{}\mdEm{Proceedings of the ACM SIGPLAN Conference on Programming
  Language Design and Implementation}% data-line={972;out\DSE-bib.bbl:60}
{}, pages 213% data-line={972;out\DSE-bib.bbl:60}
{}{\textendash}% data-line={972;out\DSE-bib.bbl:60}
{}223, 2005.%
\end{mdBibitem}%
\begin{mdBibitem}[class={bibitem},id=godefroidlm12,label={[9]\{.bibitem-label\}},elem={bibitem},cite-label={Godefroid et{\mdNbsp}al.(2012)Godefroid, Levin, and Molnar},caption={Patrice Godefroid, Michael{\mdNbsp}Y. Levin, and David{\mdNbsp}A. Molnar. \\SAGE: whitebox fuzzing for security testing.},searchterm={SAGE+whitebox+fuzzing+security+testing++Patrice+Godefroid+Michael+Levin+David+Molnar+},data-line={972;out{\textbackslash}DSE-bib.bbl:63}]%
% data-line={972;out\DSE-bib.bbl:64}
{}\mdSpan[class={bibitem-before}]{[\mdSpan[class={bibitem-label}]{9}]{\mdNbsp}{\mdNbsp}}% data-line={972;out\DSE-bib.bbl:64}
{}Patrice Godefroid, Michael% data-line={972;out\DSE-bib.bbl:64}
{}{\mdNbsp}% data-line={972;out\DSE-bib.bbl:64}
{}Y. Levin, and David% data-line={972;out\DSE-bib.bbl:64}
{}{\mdNbsp}% data-line={972;out\DSE-bib.bbl:64}
{}A. Molnar.
% data-line={972;out\DSE-bib.bbl:65}
{}\mdSpan[class={newblock}]{}% data-line={972;out\DSE-bib.bbl:65}
{} SAGE: whitebox fuzzing for security testing.
% data-line={972;out\DSE-bib.bbl:66}
{}\mdSpan[class={newblock}]{}% data-line={972;out\DSE-bib.bbl:66}
{} % data-line={972;out\DSE-bib.bbl:66}
{}\mdEm{Communications of the ACM}% data-line={972;out\DSE-bib.bbl:66}
{}, 55% data-line={972;out\DSE-bib.bbl:66}
{}\mdSpan[penalty={0}]{}% data-line={972;out\DSE-bib.bbl:66}
{} (3):% data-line={972;out\DSE-bib.bbl:66}
{}\mdSpan[penalty={0}]{}% data-line={972;out\DSE-bib.bbl:66}
{} 40% data-line={972;out\DSE-bib.bbl:66}
{}{\textendash}% data-line={972;out\DSE-bib.bbl:66}
{}44,
  2012.%
\end{mdBibitem}%
\begin{mdBibitem}[class={bibitem},id=gupta00,label={[10]\{.bibitem-label\}},elem={bibitem},cite-label={Gupta et{\mdNbsp}al.(2000)Gupta, Mathur, and Soffa},caption={Neelam Gupta, Aditya{\mdNbsp}P. Mathur, and Mary{\mdNbsp}Lou Soffa. \\Generating test data for branch coverage.},searchterm={Generating+test+data+branch+coverage++Neelam+Gupta+Aditya+Mathur+Mary+Soffa+},data-line={972;out{\textbackslash}DSE-bib.bbl:70}]%
% data-line={972;out\DSE-bib.bbl:71}
{}\mdSpan[class={bibitem-before}]{[\mdSpan[class={bibitem-label}]{10}]{\mdNbsp}{\mdNbsp}}% data-line={972;out\DSE-bib.bbl:71}
{}Neelam Gupta, Aditya% data-line={972;out\DSE-bib.bbl:71}
{}{\mdNbsp}% data-line={972;out\DSE-bib.bbl:71}
{}P. Mathur, and Mary% data-line={972;out\DSE-bib.bbl:71}
{}{\mdNbsp}% data-line={972;out\DSE-bib.bbl:71}
{}Lou Soffa.
% data-line={972;out\DSE-bib.bbl:72}
{}\mdSpan[class={newblock}]{}% data-line={972;out\DSE-bib.bbl:72}
{} Generating test data for branch coverage.
% data-line={972;out\DSE-bib.bbl:73}
{}\mdSpan[class={newblock}]{}% data-line={972;out\DSE-bib.bbl:73}
{} In % data-line={972;out\DSE-bib.bbl:73}
{}\mdEm{Proceedings of the Automate Software Engineering
  Conference}% data-line={972;out\DSE-bib.bbl:74}
{}, pages 219% data-line={972;out\DSE-bib.bbl:74}
{}{\textendash}% data-line={972;out\DSE-bib.bbl:74}
{}228, 2000.%
\end{mdBibitem}%
\begin{mdBibitem}[class={bibitem},id=king76,label={[11]\{.bibitem-label\}},elem={bibitem},cite-label={King(1976)},caption={James{\mdNbsp}C. King. \\Symbolic execution and program testing.},searchterm={Symbolic+execution+program+testing++James+King+},data-line={972;out{\textbackslash}DSE-bib.bbl:77}]%
% data-line={972;out\DSE-bib.bbl:78}
{}\mdSpan[class={bibitem-before}]{[\mdSpan[class={bibitem-label}]{11}]{\mdNbsp}{\mdNbsp}}% data-line={972;out\DSE-bib.bbl:78}
{}James% data-line={972;out\DSE-bib.bbl:78}
{}{\mdNbsp}% data-line={972;out\DSE-bib.bbl:78}
{}C. King.
% data-line={972;out\DSE-bib.bbl:79}
{}\mdSpan[class={newblock}]{}% data-line={972;out\DSE-bib.bbl:79}
{} Symbolic execution and program testing.
% data-line={972;out\DSE-bib.bbl:80}
{}\mdSpan[class={newblock}]{}% data-line={972;out\DSE-bib.bbl:80}
{} % data-line={972;out\DSE-bib.bbl:80}
{}\mdEm{Communications of the ACM}% data-line={972;out\DSE-bib.bbl:80}
{}, 19% data-line={972;out\DSE-bib.bbl:80}
{}\mdSpan[penalty={0}]{}% data-line={972;out\DSE-bib.bbl:80}
{} (7):% data-line={972;out\DSE-bib.bbl:80}
{}\mdSpan[penalty={0}]{}% data-line={972;out\DSE-bib.bbl:80}
{}
  385–394, 1976.%
\end{mdBibitem}%
\begin{mdBibitem}[class={bibitem},id=korel90,label={[12]\{.bibitem-label\}},elem={bibitem},cite-label={Korel(1990)},caption={Bogdan Korel. \\Automated software test data generation.},searchterm={Automated+software+test+data+generation++Bogdan+Korel+},data-line={972;out{\textbackslash}DSE-bib.bbl:84}]%
% data-line={972;out\DSE-bib.bbl:85}
{}\mdSpan[class={bibitem-before}]{[\mdSpan[class={bibitem-label}]{12}]{\mdNbsp}{\mdNbsp}}% data-line={972;out\DSE-bib.bbl:85}
{}Bogdan Korel.
% data-line={972;out\DSE-bib.bbl:86}
{}\mdSpan[class={newblock}]{}% data-line={972;out\DSE-bib.bbl:86}
{} Automated software test data generation.
% data-line={972;out\DSE-bib.bbl:87}
{}\mdSpan[class={newblock}]{}% data-line={972;out\DSE-bib.bbl:87}
{} % data-line={972;out\DSE-bib.bbl:87}
{}\mdEm{IEEE Transactions on Software Engineering}% data-line={972;out\DSE-bib.bbl:87}
{}, 16% data-line={972;out\DSE-bib.bbl:87}
{}\mdSpan[penalty={0}]{}% data-line={972;out\DSE-bib.bbl:87}
{}
  (8):% data-line={972;out\DSE-bib.bbl:88}
{}\mdSpan[penalty={0}]{}% data-line={972;out\DSE-bib.bbl:88}
{} 870% data-line={972;out\DSE-bib.bbl:88}
{}{\textendash}% data-line={972;out\DSE-bib.bbl:88}
{}879, 1990.%
\end{mdBibitem}%
\begin{mdBibitem}[class={bibitem},id=korel92,label={[13]\{.bibitem-label\}},elem={bibitem},cite-label={Korel(1992)},caption={Bogdan Korel. \\Dynamic method of software test data generation.},searchterm={Dynamic+method+software+test+data+generation++Bogdan+Korel+},data-line={972;out{\textbackslash}DSE-bib.bbl:91}]%
% data-line={972;out\DSE-bib.bbl:92}
{}\mdSpan[class={bibitem-before}]{[\mdSpan[class={bibitem-label}]{13}]{\mdNbsp}{\mdNbsp}}% data-line={972;out\DSE-bib.bbl:92}
{}Bogdan Korel.
% data-line={972;out\DSE-bib.bbl:93}
{}\mdSpan[class={newblock}]{}% data-line={972;out\DSE-bib.bbl:93}
{} Dynamic method of software test data generation.
% data-line={972;out\DSE-bib.bbl:94}
{}\mdSpan[class={newblock}]{}% data-line={972;out\DSE-bib.bbl:94}
{} % data-line={972;out\DSE-bib.bbl:94}
{}\mdEm{Journal of Software Testing, Verification and Reliability}% data-line={972;out\DSE-bib.bbl:94}
{},
  2% data-line={972;out\DSE-bib.bbl:95}
{}\mdSpan[penalty={0}]{}% data-line={972;out\DSE-bib.bbl:95}
{} (4):% data-line={972;out\DSE-bib.bbl:95}
{}\mdSpan[penalty={0}]{}% data-line={972;out\DSE-bib.bbl:95}
{} 203% data-line={972;out\DSE-bib.bbl:95}
{}{\textendash}% data-line={972;out\DSE-bib.bbl:95}
{}213, 1992.%
\end{mdBibitem}%
\begin{mdBibitem}[class={bibitem},id=senacav06,label={[14]\{.bibitem-label\}},elem={bibitem},cite-label={Sen and Agha(2006)},caption={Koushik Sen and Gul Agha. \\CUTE and jcute: Concolic unit testing and explicit path model-checking tools. \\In \_Proceedings of 18th Computer Aided Verification Conference\_, pages 419--423, 2006.},searchterm={Koushik+Agha+CUTE+jcute+Concolic+unit+testing+explicit+path+model+checking+tools+\_Proceedings+Computer+Aided+Verification+Conference\_+pages++},data-line={972;out{\textbackslash}DSE-bib.bbl:98}]%
% data-line={972;out\DSE-bib.bbl:99}
{}\mdSpan[class={bibitem-before}]{[\mdSpan[class={bibitem-label}]{14}]{\mdNbsp}{\mdNbsp}}% data-line={972;out\DSE-bib.bbl:99}
{}Koushik Sen and Gul Agha.
% data-line={972;out\DSE-bib.bbl:100}
{}\mdSpan[class={newblock}]{}% data-line={972;out\DSE-bib.bbl:100}
{} CUTE and jcute: Concolic unit testing and explicit path
  model-checking tools.
% data-line={972;out\DSE-bib.bbl:102}
{}\mdSpan[class={newblock}]{}% data-line={972;out\DSE-bib.bbl:102}
{} In % data-line={972;out\DSE-bib.bbl:102}
{}\mdEm{Proceedings of 18th Computer Aided Verification Conference}% data-line={972;out\DSE-bib.bbl:102}
{},
  pages 419% data-line={972;out\DSE-bib.bbl:103}
{}{\textendash}% data-line={972;out\DSE-bib.bbl:103}
{}423, 2006.%
\end{mdBibitem}%%
\end{mdBibliography}%%
\end{mdDiv}%
\begin{mdDiv}[class={madokologo,block},elem={madokologo},text-align={right},font-size={xx-small},margin-top={4em},data-line={975}]%
% data-line={976}
{}Created with{\mdNbsp}\mdA{https://www.madoko.net}{}{Madoko.net}.%
\end{mdDiv}%

\end{document}
